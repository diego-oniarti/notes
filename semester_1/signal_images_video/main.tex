\documentclass{article}

\usepackage{amsmath}

\title{Appunti Signals, Image, Video}
\author{Diego Oniarti}
\date{\today}

\begin{document}

\maketitle

\section{Definitions}
\begin{itemize}
    \item $R(t)$: Segnale rettangolare di area unitaria. Durata $\Delta T$ e ampiezza $\frac 1 {\Delta T}$. Centrato in $\frac{\Delta T}2$
    \item $\delta(t)$: impulso unitario. $\displaystyle \lim_{\Delta T\to 0}R(t)$, ma centrato in $0$.
    \item $h(t)$: risposta all'impulso di un segnale.
\end{itemize}


\section{LTI Systems}
Given a linear time invariant system, its impulse response $h(t)$, and a signal $x(t)$: \\
We can calculate the response $y(t)$ to the signal $x$ as the convolution between $x$ and $h$.
$$
y(t) = x(t) * h(t)
$$

\section{Loew level image processing}
Possiaamo modificare un'immagine nel dominio dei pixel senza introdurre rappresentazioni di livello superiore.

Lo scopo principale del low level processing è quello di migliorare l'immagine correggendo errori introdotti dal metodo di acquisizione.
\begin{itemize}
    \item \textbf{Pre-processing:} operatori che lavorano su singoli pixel
    \item \textbf{Filtering:} operatori che lavorano su un'area spaziale attorno al pixel in considerazione.
\end{itemize}

\subsection{Pre processing}
Introdurremo 3 operazioni comuni
\begin{itemize}
    \item correzione della distorsione geometrica
    \item correzione della distorsione colore
    \item monipulazione dell'istogramma
\end{itemize}

\subsubsection{Distorsione geometrica}
Effetti come \textit{fisheye} o \textit{telephoto} sono casi di distorsione geometrica. 

\paragraph{Calibration}
Questo tipo di distorsione può essere affrontato tramite \textit{calibrazione}.
Si pone un pattern noto (una scacchiera tipicamente) davanti al apparecchio di acquisizione. Possiamo poi misurare la distorsione sulla scacchiera, invertirla, e applicarla all'immagine da correggere.

È bene fare una calibrazione ogni volta che si usa una camera.

\subsubsection{Distorsione cromatica}
Un sistema può introdurre alterazioni non lineari ai componenti dei colori, modificando la percezione dei colori.

Questo piò accadere per via dell'acquisizione, del processing, o del rendering. Quindi nella catena di elaborazione potrebbero sommarsi deversi effetti di distorsione.

\section{Progetto}
È possibile fare un progetto in alternativa all'esame orale.\\
L'esame sembra la scelta migliore

\end{document}


