\documentclass{article}
\usepackage[top=2cm,bottom=2cm,left=2cm,right=3.5cm]{geometry}
\usepackage[utf8]{inputenc}
\usepackage[most]{tcolorbox}
\tcbuselibrary{skins,breakable}

\newtcolorbox{callout}[2][]{breakable,sharp corners, skin=enhancedmiddle jigsaw,parbox=false,
boxrule=0mm,leftrule=2mm,boxsep=0mm,arc=0mm,outer arc=0mm,attach title to upper,
after title={.\ }, coltitle=purple,colback=purple!10,colframe=purple, title={#2},
fonttitle=\bfseries,#1}

\newtcolorbox{warning}[2][]{breakable,sharp corners, skin=enhancedmiddle jigsaw,parbox=false,
boxrule=0mm,leftrule=2mm,boxsep=0mm,arc=0mm,outer arc=0mm,attach title to upper,
after title={!\ }, coltitle=red,colback=red!10,colframe=red, title={#2},
fonttitle=\bfseries,#1}

\newtcolorbox{esempio}[2][]{breakable,sharp corners, skin=enhancedmiddle jigsaw,parbox=false,
boxrule=0mm,leftrule=2mm,boxsep=0mm,arc=0mm,outer arc=0mm,attach title to upper,
after title={:\ }, coltitle=blue,colback=blue!10,colframe=blue, title={#2},
fonttitle=\bfseries,#1}


\usepackage[parfill]{parskip}

\usepackage{graphicx}
\usepackage{algorithm2e}

\title{Low Power Wireless Communication For The Internet Of Things}
\author{Diego Oniarti}
\date{Anno 2025-2026}

\begin{document}

\maketitle
\tableofcontents

\section{IEEE 802.15.4}
This is an old (2003) low power standard that is still being updated and improved to
this day. It is also the foundations for other standards (e.g. ZigBee and WirelessHard). \\
It does \textbf{not} provide a MAC layer and it operates on a short range (10-15\textit{m}).

IEEE 802.15.4 operates on the 2.4GHz band, overlapping with WIFI and other protocols
that can generate interference.

\section{WSN - Wireless Sensor Network}
\subsection{Applications}
There are many possible applications for WSN, some of them being:
\texttt{Wildlife monitoring}, \texttt{Glacier monitoring}, \texttt{Volcano monitoring},
\texttt{Cattle herding}, \texttt{Ocean monitoring}, \texttt{Vineyard monitoring}, 
\texttt{Cold chain monitoring}, \texttt{Rescue of avalanche victims},
\texttt{Vital sign monitoring}, \texttt{Tracking vehicles}, \texttt{Sniper localization}, \texttt{Tunnel monitoring and rescue}, \texttt{Industrial processes},
\texttt{Smart cities}, and others.

\subsection{Classification}
Wireless sensor networks can be divided by different criteria:
\begin{itemize}
    \item Goal: sense-only vs sense-and-react
    \item Interaction pattern: One-to-one, One-to-many, Many-to-many, Many-to-one
    \item Mobility: Static, Mobile nodes, Mobile sinks
    \item Space: Global, Local
    \item Time: Periodic (fixed or changing period), Event driven
\end{itemize}

\subsection {Operating Systems}
The mainstream ones are TinyOS, RIOT, FreeRTOS, Mynewt, Zephyr, and Contiki.

They offer basic run-time support for application programs but not for user interaction.

\subsection{Networking Layers}
In a classic distributed system the programmer only has to worry about the application
layer, the actual program that needs to run on the machines. In WSNs, we often have
to deal with the entire stack (i.e. MAC, Routing, transport, and application).

\paragraph{Routing}
This dictates how the messages are supposed to travel across the network. Do we want
a multi-hop mesh? Do we keep one parent or multiple ones? What traffic pattern do we
prefer?

\paragraph{MAC}
This coordinates the link-level communication. The two common approaches are CSMA
(carrier-sense multiple access) and TDMA (dime division multiple access).\\
Two major problems at this level are the need to synchronize sender and receiver, and
dealing with changing topologies.

Also, this layer has the responsibility of keeping the radio turned off as much as
possible to save energy.

\subsection{Typical requirements}
The three main requirements (or metrics) that are required by WSNs are
\begin{enumerate}
    \item Network lifetime\\
        This is defined as the time until some fraction of nodes runs out of power
    \item end-to-end reliability\\
        This is measured as the percentage of packets being received or lost
    \item end-to-end latency\\
        Some application can even give an hard boundary on this metric
\end{enumerate}
Not all applications require all three of these attributes. Monitoring systems may
only care about lifetime and a good reliability, while distributed control systems
focus more on latency and a high level of reliability.

\subsection{Typical problems}
The things that make communication hard usually are:
\begin{enumerate}
    \item Limited range, leading to multi-hop
    \item High power consumption from the radio compared to everything else
    \item Instability of the links, both over time and space
\end{enumerate}

\section{MAC - Medium Access Control}
As the name suggests, the MAC regulates which node has access to the shared channel
(or medium).

Medium access control for wireless sensor networks is slightly different from the
one for general purpose systems. It trades performance for energy consumption (within
reason). Throughput, fairness, and often latency are of secondary importance.

The size of the data packets being sent is often small (in the order of bytes or
tenths of bytes even), so the size of the headers has more impact in percentage. This
can create a significant overhead, also contributed to by control messages like RTS
and CTS.

MAC protocols in some cases (e.g. event-based applications) must account for bursty
data patterns, in which a node might remain silent for a long time and sporadically
send a burst of data.

\subsection{CSMA - Carrier Sense vs Collision Avoidance}\label{sub:CSMA}
\paragraph{Carrier Sense}
Listen on the channel to check whether or not there is a communication going on
already. If that is not the case you can transmit.\\
Collisions can still occur since the time to start a transmission is not zero. In
that case the nodes back-off by a random amount and retry the transmission.

Another problem with this approach is that of the \textit{hidden terminal}. This is
a common and well known problem in which two nodes $A$ and $B$ are in range of node
$C$ but not of each other. When $A$ and $B$ sense the carrier they see it as being
clear (since they are outside of each other's range) so they both transmit to $C$,
which gets two interfering signals.

\paragraph{Collision Avoidance}
Add some additional control messages (i.e. \texttt{request-to-send} and \texttt{clear-to-send})
to address the hidden terminal problem.\\
If a node $A$ wants to transmit it sends a \texttt{request-to-send} tagged with its id. $C$
receives it and broadcasts to its neighbors a \texttt{clear-to-send} with $A$'s id.
Now $A$ knows it can send it's original message, while $C$ knows it can't send anything
because it received a \texttt{clear-to-send} with someone else's id ($A$'s).

\subsection{TDMA - Time Division Multiple Access}\label{sub:TDMA}
Time divided in \textit{frames} which are further divided in \textit{slots}.
A typical frame layout has slots for:
\begin{itemize}
    \item traffic control: meta-information from the base station to the nodes
    \item downlink: to send application data from the base station to the nodes
    \item uplink: to send application data from the nodes to the base station
    \item contention period: to allow new nodes to join the system
\end{itemize}


\subsection{Slotted access}\label{sec:slotted_access}
\begin{wrapfigure}[5]{R}{120px}
    \centering
    \includegraphics[width=100px]{images/slotted.png}
    \caption{Slotted access}
\end{wrapfigure}
A sort of mix between CSMA and TDMA. All nodes wake up and go to sleep together to
ensure synchronization. The awake time is short and the nodes must in some way decide
how to handle pending packets during this contention period. This is prone to collision.

\subsection{S-MAC}
This is an improvement on \hyperref[sec:slotted_access]{slotted access}, aiming at
reducing collisions.\\
This is done by roughly synchronizing the wake up clocks of just neighbors, not all
the nodes. The active period is divided in two parts: a portion to exchange
synchronization messages, and one to exchange RTS, CTS, and data messages.

Furthermore, SYNC messages may only be sent in some awake periods (1 every N for
example) to reduce collisions.

This MAC protocol does not guarantee fairness or low latency.

\subsection{D-MAC}
The \textit{Data-gathering} MAC breaks the ISO/OSI layer abstraction, utilizing routing
information to inform the MAC.\\
The nodes are activated with a slight delay across layers, with the bottom two
activating first. Each subsequent layer is activated slightly after the one below.

This approach reduces both latency and contention, but it only really works well
in static networks.

\subsection{B-MAC / LPL}\label{sub:LPL}
\textit{Low Power Listening} is used to limit the power consumption on the listener
while moving the blunt of the work on the transmitter.
\begin{figure}[h!]
    \centering
    \includegraphics[width=0.8\linewidth]{images/B-MAC.png}
    \caption{fig:B-MAC}
\end{figure}
The receiver sends a long preamble (at least longer than the sleep interval) followed
by the message.
The receiver periodically checks the channel. If it sees the preamble being
transmitted it waits for it to end and listens to the message.\\
The preamble also contains header information, like the message target. So other
listeners that wake up can see the preamble destined to another node and resume sleeping.

A long preamble consumes a lot of energy in the sender, while a short preamble requires
the listener to wake up more often.
Long preambles also make it more likely for collisions between transmitters.

\subsection{CCA - Clear Channel Assessment}
Checks if the channel is clear for transmission. Must be able to tell if what the
node sees on the channel is noise or the transmission from another node.

\subsection{LPP - Lop Power Probing}
\begin{wrapfigure}[5]{R}{0.5\textwidth}
    \centering
    \includegraphics[width=0.8\linewidth]{images/LPP.png}
    \caption{fig:LPP}
\end{wrapfigure}
This is the "opposite" of \hyperref[sub:LPL]{LPL}. The receiver sends out periodic
beacons (Ready to receive). When a sender wants to transmit it listens for the
beacon and then sends its message.

With the long preamble on the sender being replaced by a long listen, this reduces
both power consumption and the risk of collision.

\subsection{A-MAC}
Radio probes trigger ACK non-destructively thanks to tight timing.

\subsection{Z-MAC}
\textit{Zebra MAC} ``switches" between \hyperref[sub:CSMA]{CSMA} and
\hyperref[sub:TDMA]{TDMA} depending on the contention on the channel in each given
time.

The two modes formally are HCL (High Contention Level) and LCL (Low Contention Level).\\
In High Contention the transmitter only sends messages during its allotted time slot (TDMA),
while in Low Contention it checks the medium for ongoing transmissions and waist a
random backoff if it hears any (CSMA).

\section{Routing}
In normal contexts, routing would be an invisible part of the network protocol to the
developer, simply seeing a socket as an interface to the network. In WSN however, the
layers are blurred, requiring the developer to worry about routing as well.

\subsection{LEACH}
\textit{Low-Energy Adaptive Clustering Hierarchy} is a routing protocol organized in
\textit{rounds}, each one further divided into two \textit{phases}.
\paragraph {Setup Phase}
\begin{itemize}
    \item each node has a random probability of becoming a Cluster Head (CH), and it broadcasts an advertisement.
    \item Normal nodes get the advertisements and chose an head, effectively joining its cluster.
    \item The CH defines a TDMA schedule to communicate with the nodes in its cluster
        and it broadcasts it to them.
\end{itemize}

\paragraph{Steady Phase}
Normal nodes send their data to their local head. The head collects messages from the
nodes, merges them in one bigger message, and sends this fused message to the base
station.

\subsection{Directed Diffusion}
Data-centric routing for many-to-one communication with multiple topics.\\
This approach supports data in \texttt{key:value} format, with multiple sinks that
may be \textbf{interested} in different attributes.

Each sink floods the system with its interest, allowing the nodes to build a reverse
path tree for each interest (gradients). When a sensor has to report some data in
just sends it along the ``gradient" for that specific key/topic.

The trees must be periodically rebuilt/reinforced through periodic flooding. Another
important decision is the metric with which a node chooses its parent. Some common
options are
\begin{itemize}
    \item hop count
    \item link quality
    \item workload and residual every
    \item application information
\end{itemize}

\paragraph{Workload} can greatly affect the lifetime of a node. Nodes closer to the
sink tend to route messages from more nodes, leading to higher battery consumption.\\
One solution might be load balancing, making different nodes take turns based on their
estimated lifetime, but this can be complex to implement. Another solution can be to
simply instal bigger batteries on the nodes closest to the sink. Lastly, we can use
in-network aggregation to reduce the number of communications.

\subsection{MintRoute}
This is the routing protocol in TinyOS 1. It periodically sends out beacons from the
sink that are then retransmitted by the lower nodes. The metric is
\[ m = m_p + \frac 1 {LQI_p^3} \]
where $m_p$ is the metric transmitted by the parent and $LQI_p$ is the Link Quality
between the node and the parent.

\subsection{CTP - Collection Tree Protocol}\label{sub:CTP}
The routing protocol in TinyOS 2. It uses ETX instead of LQI to determine which
parent to follow.

\subsection{MUSTER}
Routing protocol used for multi-source multi-sink routing. Usually in this scenario
each sink would have its own dedicated tree, but this solution is very expensive.

MUSTER aims at overlapping as much of the trees as possible, and collect multiple
readings in the same message, to reduce energy consumption on heavy forwarders.

\subsection{RPL}
A quite heavy protocol that supports \texttt{multipoint-to-point},
\texttt{point-to-point}, and \texttt{point-to-point} communication. It stores the
routes as a DODAG (Destination Oriented Directed Acyclic Graph) and keeps track of
an objective function metric similar to the \hyperref[sub:CTP]{CTP} one.

The protocol can operate in two modes: Storing mode, and non-storing mode.
\paragraph{Storing Mode:} Every node stores the routing table for its own sub-DODAG.
This can take up a lot of memory on the nodes.
\paragraph{Non-storing Mode:} Only the sink keeps a routing table of all the nodes.
To collect information the messages need to accumulate the path that they have taken,
leading to possibly very large messages.

\paragraph{Point-to-point routing}

\begin{wrapfigure}{R}{.5\textwidth}
    \centering
    \includegraphics[width=0.8\linewidth]{images/RPL_ptp.png}
    \caption{fig:RPLptp}
\end{wrapfigure}
differs slightly in the two modes. In storing mode a message from node $A$ to node
$B$ only needs to travel up to the lowest common parent between the two. In
non-storing mode the messages have to travel from $A$ up to the sink, and from the
sink down to $B$.

\paragraph{Problems} This protocol is made quite heavy by its need to support IPv6,
which adds to the size of the packets, along with the memory occupied by the routing
tables.

\subsection{Orchestra}
Protocol used in industrial scenarios that tries to be both flexible and reliable.
It does this by autonomously generating TSCH schedules informed by the routing state.

The TSCH aspect makes it reliable, while the RPL information makes it more flexible.

\paragraph{3 slot types}
\begin{enumerate}
    \item Rendez-vous: this one is for discovery and routing. It is contention based.
    \item Receiver-based shared: This one is contention based too. Each node gets a
        single receive slot and it's used for unicast.
    \item sender-based shared (/dedicated): Also for unicast. Each node gets several
        receive slots, potentially wasting energy but reducing contention.
\end{enumerate}

\subsection{Opportunistic forwarding}
Routing technique based on broadcasting instead of unicast. Each node broadcasts its
messages, and if someone else can receive them, they forward them further. If a node
receives a message and sees another node already forwarding it, it drops it.

\subsection{ORW - Opportunistic Routing in WSN}

\begin{wrapfigure}{R}{.5\textwidth}
    \centering
    \includegraphics[width=0.8\linewidth]{images/ORW.png}
    \caption{fig:ORW}
\end{wrapfigure}
It still relies on a routing metric, except this time it is EDC (Expected Duty
Cycled Wakeups).

\subsection{Glossy}
This protocol broadcasts messages without building any topology first. The messages
are simply broadcast from one node, then all the nodes that received the message
broadcast it as well. This protocol requires \textit{accurate network-wide synchronization},
so that the messages may collide non destructively.

This makes the propagation of messages fast and reliable (due to the inherit
redundancy), but requires synchronization assumptions. It is also tolerant of mobility.

\section{Data Prediction}
In some circumstances we can reduce the amount of data being sent over the air by
only sending enough to build a \textit{model}. The aggregator can simply assume data
will keep coming in conformance to that model, and the sensors only need to send
outliers.

\subsection{DBP - Derivative Based Prediction}
This approach works well with linear data (or data that is linearisable over short
spans of time). The model that is built and predicted only requires a point and an
angle, so it is quite simple. Determining of a data point is in the error margin is
also computationally simple, since we only need the distance from the point to the
line.

We set a \textit{value tolerance} and a \textit{time tolerance}. We start sending data
again when we get values outside of the tolerance for too long of a time.

One drawback with this approach is the face that the data being transmitted is
\begin{itemize}
    \item unpredictable
    \item bursty
    \item critical. Since losing a message has a great impact on the system
\end{itemize}
All these properties are well server by the \hyperref[sub:glossy]{glossy} stack.

\subsection{Glossy}\label{sub:glossy}
Glossy is a network stack designed to collect sparse and aperiodic data. This type
of traffic can be generated by data prediction algorithms or by networks used for
event handling and command issuing instead of data collection.

The system is built atop the glossy floods. The nodes flood their data and the receiver
floods its acknowledgment. If a node receives the ack to its own message it stops
transmitting. If it receives the ack to the message of another node it transmits its
own message again.

\section{LPWAN - Low Power Wide Area Network}
Low power WANs can serve large areas (in the tens of kilometers range) for multiple
years, but with very \textit{low data throughput} and \textit{high latency}.\\
They often use simple star topologies, where many nodes can be served by the same
base station. This makes the network stack more simple, not requiring multi-hop, and
puts the heavy work on the base station instead of the devices.

\paragraph{Narrowband}
Transmitting over a narrow frequency band allows for more links to efficiently share
the spectrum, and makes noise have a smaller impact on the signal's power. This also
allows for simpler receivers and antennas.

\textit{Ultra-narrow band} signals take this to an extreme, increasing latency significantly
as the signal is kept in the hundreds of hertz range.

\paragraph{Spread Spectrum} takes the opposite approach to narrowband, spreading the
signal over a much wider band of the spectrum. This makes the signal appear almost
like noise, making it less susceptible to jamming and actual noise, but requiring a
more sophisticated decoder on the receiver's end.

The two common implementations of this are DSSS (Direct Sequence Spread Spectrum) and
CSS (Chirp Spread Spectrum).

\section{Localization}

\subsection{Localization vs Detection}
Proximity detection only requires to measure or determine the distance between two
actors, while localization needs to provide $(x,y,z)$ coordinates of a actor in a
given coordinate space.

\subsection{Anchors}
Localization techniques assume the existence of some reference point(s) whose
location is known. These beacons or anchors are used to build the coordinate system
the localization will take place in.

\subsection{Device-based vs Device-free}
Device-based localization requires the actor (usually a person, animal, or robot) be
equipped with a device that can communicate with the anchors.

Device-free localization is usually based on the way an actor itself disturbs the
wireless communication between beacons. Radar and lidar are also device free localization
methods.

\subsection{RF-based technologies}
\paragraph{RF-id} is cheap and passive, meaning it does not require a battery. It's
range is very poor, some times requiring direct contact.

\paragraph{WiFi} is ubiquitous and supports both ranging and communication, but it
consumes a lot of power.

\paragraph{Bluetooth} is ubiquitous in some contexts, can transmit data as well as
ranging, and consumes less energy than WiFi, but its localization is rougher and its
range is shorter.

\paragraph{Ultra-wide band} is accurate and operates over a larger range, but it
requires dedicated infrastructure to work.

\subsection{Computing the position}
The position of an actor is usually calculated starting from its distance from the
anchors. If we plug this value into the equation of a circle $(x_i-x)^2+(y_i-y)^2=d^2$ (or
into that of a sphere depending on context) we can build a system of equations. This
is then solved with the least squares method to get the actor's position.
\[ residue = \frac{\sum_{i=1}^n\sqrt{(x_i-\hat x)^2+(y_i-\hat y)^2}-d_i}{n} \]
If the residue is too large the measurement is discarded.

\end{document}
