\documentclass{article}

\newcommand{\na}[0]{\ensuremath {\overset{N}{\rightarrow}}}
\newcommand{\rl}[3]{\inference{#1}{#2}\text{ #3}}
\newcommand{\bop}[0]{\ensuremath\oplus}
\newcommand{\appl}[2]{\ensuremath(#1)\ #2}
\newcommand{\st}[3][]{\ensuremath{\displaystyle\frac{#3\hfill}{#2\hfill} \text{#1}}}
\newcommand{\N}{\ensuremath \mathbb N}
\newcommand{\lam}[2]{\ensuremath{\lambda#1.#2}}
\usepackage{pdflscape}
\usepackage[utf8]{inputenc}
\usepackage{amsmath}
\usepackage{amssymb}
\usepackage{hyperref}
\usepackage{multirow}
\usepackage[most]{tcolorbox}
\tcbuselibrary{skins,breakable}

\newtcolorbox{callout}[2][]{breakable,sharp corners, skin=enhancedmiddle jigsaw,parbox=false,
boxrule=0mm,leftrule=2mm,boxsep=0mm,arc=0mm,outer arc=0mm,attach title to upper,
after title={.\ }, coltitle=purple,colback=purple!10,colframe=purple, title={#2},
fonttitle=\bfseries,#1}

\newtcolorbox{warning}[2][]{breakable,sharp corners, skin=enhancedmiddle jigsaw,parbox=false,
boxrule=0mm,leftrule=2mm,boxsep=0mm,arc=0mm,outer arc=0mm,attach title to upper,
after title={!\ }, coltitle=red,colback=red!10,colframe=red, title={#2},
fonttitle=\bfseries,#1}

\newtcolorbox{esempio}[2][]{breakable,sharp corners, skin=enhancedmiddle jigsaw,parbox=false,
boxrule=0mm,leftrule=2mm,boxsep=0mm,arc=0mm,outer arc=0mm,attach title to upper,
after title={:\ }, coltitle=blue,colback=blue!10,colframe=blue, title={#2},
fonttitle=\bfseries,#1}

\newcommand{\na}[0]{\ensuremath {\overset{N}{\rightarrow}}}
\newcommand{\rl}[3]{\inference{#1}{#2}\text{ #3}}
\newcommand{\bop}[0]{\ensuremath\oplus}
\newcommand{\appl}[2]{\ensuremath(#1)\ #2}
\newcommand{\st}[3][]{\ensuremath{\displaystyle\frac{#3\hfill}{#2\hfill} \text{#1}}}
\newcommand{\N}{\ensuremath \mathbb N}
\newcommand{\I}{\ensuremath \mathbb I}
\newcommand{\lam}[2]{\ensuremath{\lambda#1.#2}}
\newcommand{\inl}[0]{\ensuremath{\ inl\ }}
\newcommand{\inr}[0]{\ensuremath{\ inr\ }}
\newcommand{\case}[3]{\ensuremath{\text{case}#1\ \text{of}\ \left|\begin{aligned}& #2\\ & #3\end{aligned}\right.}}
\newcommand{\Da}[0]{\ensuremath{\Downarrow}}
\newcommand{\while}[2]{\ensuremath{\text{while }#1\text{ do }#2\text{ end}}}
\newcommand{\for}[3]{\ensuremath{\text{for }i=#1\text{ to }#2\text{ do }#3\text{ end}}}
\newcommand{\mE}[0]{\ensuremath{\mathbb{E}}}
\newcommand{\pair}[1]{\ensuremath{\langle#1\rangle}}
\newcommand{\V}{\ensuremath{\mathcal{V}}}
\newcommand{\cE}{\ensuremath{\mathcal{E}}}
\newcommand{\cD}{\ensuremath{\mathcal{D}}}
\newcommand{\cF}{\ensuremath{\mathcal{F}}}
\newcommand{\IF}[0]{\ensuremath {\text{ if }}}
\newcommand{\THEN}[0]{\ensuremath {\text{ then }}}
\newcommand{\ELSE}[0]{\ensuremath {\text{ else }}}
\newcommand{\AND}[0]{\ensuremath {\text{ and }}}
\newcommand{\OR}[0]{\ensuremath {\text{ or }}}
\newcommand{\unpack}[3]{\ensuremath{\text{unpack } #1 \text{ as }\langle #2 \rangle\text{ in }#3}}
\newcommand{\pack}[2]{\ensuremath{\text{pack } \pair{#1} \text{ as } #2 }}
\newcommand{\te}[1]{\text{#1}}
\newcommand{\ls}[0]{\ensuremath{\leadsto^{*}}}
\newcommand{\LET}[0]{\ensuremath{\text{ let }}}
\newcommand{\TIN}[0]{\ensuremath{\text{ in }}}
\newcommand{\NEW}[0]{\ensuremath{\text{ new }}}


\definecolor{dkgreen}{rgb}{0,0.6,0}
\definecolor{gray}{rgb}{0.5,0.5,0.5}
\definecolor{mauve}{rgb}{0.58,0,0.82}

\lstset{frame=tb,
    language=Java,
    aboveskip=3mm,
    belowskip=3mm,
    showstringspaces=false,
    columns=flexible,
    basicstyle={\small\ttfamily},
    numbers=none,
    numberstyle=\tiny\color{gray},
    keywordstyle=\color{blue},
    commentstyle=\color{dkgreen},
    stringstyle=\color{mauve},
    breaklines=true,
    breakatwhitespace=true,
    tabsize=3
}

\title{Appunti Concurrency}
\author{Diego Oniarti}
\date{Anno 2024-2025}

\begin{document}

\maketitle
\tableofcontents

\section{Goals of the course}
\begin{itemize}
    \item get a general understnding of the issues
    \item develop the ability to reason about
        \begin{itemize}
            \item what can go wrong under interference
            \item what can be safely made mode efficient by concurrent cooperation
        \end{itemize}
\end{itemize}
Più nello specifico vedremo algoritmi di sincronizazione, tecniche per ragionare formalmente sulla concorrenza, e i fondamentali:
cos'è fattibile, cosa non lo è, e perché.

Il libro che segue la Quaglia è "\textit{The art of multiprocessor programming, revised print}".

\subsection{Modalità d'esame}
Orale obbligatorio e progetto opzionale. Il progetto rende l'orale più semplice.
Il progetto andrebbe fatto prima di Natale.


\section{Concurrent vs Sequential}
Un programma sequenziale deve occiparsi solo della "\textbf{safety}". Quindi assicurarsi che l'output sia corretto per un dato input. 

Un programma concorrente, invece, deve occuparsi sia della safety che della "\textbf{liveness}". Questo è un'aspetto che non ha eguali nella programmazione sequenziale. Con liveness definiamo la capacità di un programma di eseguire correttamente a tempo indeterminato senza supervisione dall'utente (credo). \\
Pensare alla safety nella programmazione concorrente è molto più complesso che nella programmazione sequenziale, e deve essere garantita a prescindere dall'ordine di esecuzione dei thread.

\section{Underlying Model}
Assumiamo:
\begin{itemize}
    \item Thread Multipli (processi)
    \item Singola memoria condivisa
    \item Gli oggetti risiedono in memoria
    \item Delay asincroni inprevedibili
\end{itemize}

\section{Road Map}
Inizieremo parlando di principi e use-cases dopo.

\section{Esempio numeri primi}
Immaginiamo di voler usare 10 processi per calcolare i numeri primi da $0$ a $10^10$. \\
L'intuizione naive sarebbe quella secondo cui questo è 10 volte più veloce rispetto a farlo con un processore singolo. Ma raramente abbiamo questo tipo di improvement.

\paragraph{Primo approccio}
Il promo approccio potrebbe essere quello di dividere il range in $10$ e associare ad ogni thread uno dei $10$ range in cui calcolare i numeri primi.

Questa opzione non è buona per il fatto che il task dell'ultimo thread è molto più complesso del task del primo. Questo è un problema di "load balancing".

\paragraph{Metodo più dinamico}
Usare un contatore condiviso. Ogni processo prende il primo numero libero che trova e testa se è primo. 

Il problema ora è la gestione del counter condiviso. Se più thread tentano di accedere al counter in modo non atomico potrebbero sovrascriverlo con valori sbagliati o leggere valori non aggiornati.

\section{Semafori}
Questo non è safe, in quanto la lettura e la scrittura della variabile \texttt{temp} non sono sincrone.
\begin{lstlisting}
public class Counter {
    private long value;

    public long getAndIncrement() {
        temp = value;
        value = temp+1;
        return temp;
    }
}
\end{lstlisting}

Affrontare la safety è diverso in diversi linguaggi. In Java possiamo usare la keywork "synchronized"
\begin{lstlisting}
public class Counter {
    private long value;

    public long getAndIncrement() {
        synchronized {
            temp = value;
            value = temp+1;
        }
        return temp;
    }
}
\end{lstlisting}

\begin{callout}{Safety and Liveness}
    Safety: Niente di brutto deve succedere. \\
    Liveness: Prima o poi deve succedere qualcosa di bello.
\end{callout}

\section{Amdahl's Law}
Questa formula ci indica quanto più veloce un programma può essere reso utilizzando concorrenza:
$$
speedup = \frac{1}{1-p+\frac p n}
$$
dove $p$ è la percentuale di codice che può essere parallelizzata e $n$ è il numero di processi.

\section{Mutual Exclusion}
\subsection{Time And Events}
Un evento $a_0$ di un thread $A$ è \textit{istantaneo} e \textit{non} esistono eventi simultanei.

Possiamo vedere un thread come una \textit{macchina a stati}.
\paragraph{State}
Differenziamo lo stato di un thread dallo stato del sistema. \\
State: \begin{itemize}
    \item program counter
    \item local variables
\end{itemize}
System state: \begin{itemize}
    \item shared variables
    \item thread states
\end{itemize}

\paragraph{Intervalli}
Un intervallo $A_0=(a_0,a_1)$ è il tempo tra gli eventi $a_0$ e $a_1$.

\subsubsection{Precedenza}
Possiamo dire che un intervallo \textit{precede} un'altro intervallo ($a_1 < b_0$)
\paragraph{Irreflexive} Never true that $I\to I$
\paragraph{Antisymmetric} $I\to J \implies I\not\to J$
\paragraph{Transitive} $I\to J \text{ and } J \to K \implies I\to K$
\paragraph{Partial} 

\subsubsection{Eventi Ripetuti}
$a_0^k$ indica la $k$-esima occorrenza dell'evento $a_0$, qual'ora esso si ripeta. \\
Lo stesso si può fare con l'intervallo $A_0^k$.

\subsection{Locks}
Definiamo due metodi \textit{lock} e \textit{unlock} che incapsulreanno la sezione critica.
\begin{lstlisting}
public class Counter {
    private long value;
    private Lock mylock;
    public long getAndIncrement() {
        // Acquisisci il lock
        mylock.lock();
        try {
            // Sezione critica
            int temp = value;
            value = value+1;
        } finally {
            // Rilascia il lock a ogni costo, anche in caso di errore
            mylock.unlock();
        }
        return temp;
    }
}
\end{lstlisting}
\begin{callout}{java}
    Usiamo Java perché è il linguaggio usato nel libro di testo. Non per altri motivi.
\end{callout}

\subsection{Deadlock-Free}
Se un qualche thread chiama \textit{lock()}, primo a poi un qualsiasi thread deve acquisire il lock.
\subsection{Starvation-Free}
Se un qualche thread chiama \textit{lock()}, prima o poi deve acquisire il lock.

\subsection{Soluzioni per due/n thread}
Inizieremo con soluzioni che risolvono il problema con due thread prima di muoverci a soluzioni più generali e complesse.\\
Indicheremo con $i$ il thread corrente e $j$ l'altro.

\subsubsection{LockOne}
\begin{lstlisting}
class LockOne implements Lock {
    private boolean[] flag = new boolean[2];
    public void lock() {
        // Setta la flag e aspetta che l'altra chiuda
        flag[i] = true;
        while (flag[j]) {}
    }
    public void unlock() {
        flag[i] = false;
    }
}
\end{lstlisting}

Questa prima implementazione primitiva garantisce mutual exclusion ma non è deadlock-free. \\
Se i due thread settano la flag allo stesso momento entrano in deadlock.

\subsubsection{LockTwo}
\begin{lstlisting}
class LockTwo implements Lock {
    private int victim;
    public void lock() {
        victim = i;
        while (victim==i) {}
    }
    public void unlock() {}
}
\end{lstlisting}

\subsubsection{Peterson's Algorithm}
\begin{lstlisting}
public void lock() {
    flag[i] = true;
    victim = i;
    while (flag[j] && victim==i) {};
}
public void unlock() {
    flag[i] = false;
}
\end{lstlisting}
Questo algoritmo risolve sia mutual exclusion che starvation.

L'ordine delle operazioni è importante. Se cambiassimo l'ordine delle attribuzioni di \texttt{flag[i]} e \texttt{victim} rompiamo già la mutua esclusione.

L'algoritmo può essere visto come un "filtro". Dove settare la flag indica il desiderio di un thread di entrare nella CS, e la variabile victim funge da discriminatore.

\subsubsection{Filter Algorithm}
L'algoritmo di Peterson può essere generalizzato a più thread con una soluzione a "livelli"
\begin{lstlisting}
int[] level;
int[] victim;

public Filter(int n) {
    public void lock() {
        for (int L=1, L<n; L++) {
            level[i] = L;
            victim[L] = i;
            while (((exists k!= i) level[k]>=L) && victim[L]==i) {};
        }
    }
    public void unlock() {

    }
}
\end{lstlisting}
Questo algoritmo è starbation free e garantisce mutua esclusione. Non è buono a livello di fairness però.

\subsection{Bounded Waiting}
Possiamo dividere un metodo \textit{lock} in due parti. Il "\textit{doorway interval}" e il "\textit{waiting intervall}". 
Per garantire un qualche livello di fairnes dobbiamo imporre un upper bound alla durata dell'intervallo di waiting.

Un thread non può "superare" un altro più di $r$ volte. $r$ è un superparametro imposto dallo sviluppatore.
\begin{callout}{superare}
    Diciamo che un thread $A$ ne "supera" uno $B$ quando $A$ entra ed esce dalla sezione critica mentre $B$ rimane bloccato nel waiting della lock.
\end{callout}

\subsubsection{Bakery Algorithm}
L'algoritmo del fornaio assicura un ordine FIFO per i thread in attesa.
\begin{lstlisting}
boolean[] flage;
Lavel[] lavel;
public void lock() {
    // doorway
    flag[i] = true;
    label[i] = max(label[0],...,label[n-1])+1;
    // waiting
    while (forall k != i:
        (flag[k] 
        && 
        (label[k],k) << (label[i],i)
        )
    ) {};
}
\end{lstlisting}
$(lavel[a],a) << (label[b],b)$ ordine lessicografico.

L'algoritmo a livello teorico funziona perfettamente. L'implementazione è però difficile o impossibile.
\begin{enumerate}
    \item Le lable esplodono a numeri infiniti, causando overflow
    \item Ci possono essere conflitti durante il calcolo di max
    \item Ci possono essere conflitti durante il calcolo del while
\end{enumerate}
Gli ultimi due punti possono essere risolti dall'ordine lessicografico ma il primo no.

\section{Precedence Graph}
One can construct a 
\begin{itemize}
    \item wait-free (no mutual exclusion)
    \item concurrent
    \item timestamping system
    \item that never overflows
\end{itemize}
Un grafo dove an edge from $x$ to $y$: \begin{itemize}
    \item $x$ is later timestamp
    \item missing
    \item missing
\end{itemize}
Quando usiamo \textit{grafi di precedenza} non facciamo ricorso alla transitività per capire "chi viene prima", ma sono gli archi.

\section{Numero minimo di variabili}
Non esiste un algoritmo che assicuri mutua esclusione tra $n$ thread con meno di $n$ variabili.

Assumiamo per assurdo che esista un algoritmo di questo tipo. Possiamo poi dimostrare che un algoritmo di questo tipo rompe la mutua esclusione (provando la tesi per assurdo).

Definiamo dei termini.
\begin{itemize}
    \item $S$ è \textit{idle} se tutti i thread sono nella loro reminder region.
    \item $S\sim_A S'$ due stati sono \textit{indistinguibili} se
        \begin{itemize}
            \item Le variabili locali sono uguali
            \item Le variabili globali sono uguali
        \end{itemize}
\end{itemize}

Ora dei fatti:
\begin{enumerate}
    \item Any thread $A$ tunning solo from S, with either $S$ idle or $S\sim I$ and $I$ idle, can reach CS
    \item Any thread $A$ that from an idle state reaches CS running solo must write something in shared memory before getting in.
\end{enumerate}

\section{Tight bound: n bite for n threads}

\section{Hoare Logic}
La \textit{Hoare Logic} era inizialmente utilizzata per ragionare sulla correttezza di programmi sequenziali. È un tipo di analisi sintattica (quindi statica).

\begin{itemize}
    \item At the basis of all deductive verification techniques \\
        Anche PVS usato dall'NSA si basa sulla Hoare Logic
    \item Used to specify properties of squential imperative programs by means of triples
        \begin{align*}
            \{p\} S \{q\}
        \end{align*}
        where 
        \begin {itemize}
            \item $S$ is a statement of the language
            \item $p$ is the \textit{pre-condition}, an assertion about a relevant property which the state holds before $S$ is exectuted
            \item $q$ is a \textit{post-condition}, an assertion that holds after $S$
        \end{itemize}
        In altre parole
        \begin{itemize}
            \item if $S$ starts execution in a state satisfying $p$
            \item and if the execution of $S$ terminates
            \item Then the state reached after the execution of $S$ satisfyes $q$
        \end{itemize}
\end{itemize}

\subsection{Language of assertions}
Assertions are formulas in a propositional loguc
\begin{align*}
    A ::=& A\ and\ A \\
         |& A\ or\ A \\
         |& true \\
         |& false \\
         & ... \\
    S ::= & skip\ |x:=E | S;S| if B then S else S | while (B) S
\end{align*}

Evert state sarisfies $true$ and no state sarisfies $false$. Henche fpr every $p$ and for every $S$
\begin{align*}
    \{p\} S \{true\}
\end{align*}
is a valid triple. and $\{false\} S \{p\}$ is a valid triple.

\begin{esempio}{es}
    \[\begin{array}{cl}
        \{x=5\}x:=x+1\{x=6\} & \text{è valido} \\
        \{x=5\ and\ y=2\}x:=x+1\{x=6\ and\ y=3\} & \text{non è valido} \\
        \{x=5\ and\ y=2\}x:=x+1\{x=6\ or\ y=3\} & \text{è valido} \\
    \end{array}\]
\end{esempio}

\subsection{Proof System}
A proof system is defines so that: if the proof system accepts $\{p\}S\{q\}$, $\{p\}S\{q\}$ is valid.
\subsubsection{Proof outline}
Un modo più compatto di scrivere derivazioni nel proof system per non avere alberi enormi.

\begin{esempio}{es}
\[\begin{array}{c}
    \{x+1\} \\
    \{x+1=y+1\} \\
    x:=x+1 \\
    \{x=y+1\} \\
    y:=y+1 \\
    \{x=y\}
\end{array}\] 
\end{esempio}

\begin{esempio}{es}
    \begin{align*}
        \{x=5\ &and\ y=2\} \\
        \{x&=5\} \\
        \{x+1&=6\} \\
        x&:=x+1 \\
        \{x&=6\} \\
        \{x=6\ &or\ y=3\}
    \end{align*}
\end{esempio}

\begin{esempio}{es}
    \[\begin{array}{c}
        \{x=3\} \\
        \{x>2\} \\
        \{x+1>3\} \\
        x:=x+1; \\
        \{x>3\} \\
        \{x+2>5\} \\
        x:=x+2; \\
        \{x>5\}
    \end{array}\] 
\end{esempio}

\section{2024-10-31}
\begin{itemize}
    \item test and set: Molti cache miss
    \item test and test and set: Meno cache miss
    \item exponential backoff: Meno chase miss ancora ma dipende molto da\\ parametri dati
\end{itemize}

\section{Linearizability}
Each method should take effect instantaneously between invasion an response evenest.

A linearisable object is one whose possible executions are linearisable.

Split method Calls into two events:
\begin{itemize}
    \item invocation \\
        q.enq(x)
    \item response \\
        q.enq(x) returns void / x / empty
\end{itemize}


\end{document}

