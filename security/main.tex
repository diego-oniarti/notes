\documentclass{article}

\usepackage[utf8]{inputenc}
\usepackage{amsmath}
\usepackage[most]{tcolorbox}
\tcbuselibrary{skins,breakable}

\newtcolorbox{callout}[2][]{breakable,sharp corners, skin=enhancedmiddle jigsaw,parbox=false,
boxrule=0mm,leftrule=2mm,boxsep=0mm,arc=0mm,outer arc=0mm,attach title to upper,
after title={.\ }, coltitle=purple,colback=purple!10,colframe=purple, title={#2},
fonttitle=\bfseries,#1}

\newtcolorbox{warning}[2][]{breakable,sharp corners, skin=enhancedmiddle jigsaw,parbox=false,
boxrule=0mm,leftrule=2mm,boxsep=0mm,arc=0mm,outer arc=0mm,attach title to upper,
after title={!\ }, coltitle=red,colback=red!10,colframe=red, title={#2},
fonttitle=\bfseries,#1}

\newtcolorbox{esempio}[2][]{breakable,sharp corners, skin=enhancedmiddle jigsaw,parbox=false,
boxrule=0mm,leftrule=2mm,boxsep=0mm,arc=0mm,outer arc=0mm,attach title to upper,
after title={:\ }, coltitle=blue,colback=blue!10,colframe=blue, title={#2},
fonttitle=\bfseries,#1}



\title{Appunti Security Testing}
\author{Diego Oniarti}
\date{Anno 2024-2025}

\begin{document}

\maketitle
\tableofcontents

\section{Heap vs Stak}
Ciò che allochi sulla heap non viene deallocato quando la funzione finisce. Quello che viene allocato dalla stack viene distrutto a fine lezione. Questo lo ha detto anche Roveri nel primo semestre di triennio.

\section{Classification and groups of bugs / errors}
There are many ways of dividing errors into taxonomies. The major ones consist in ranking them on a certain metric. This can be:
\begin{itemize}
    \item \textbf{severity:} how much of an impact the bug can have on the system
    \item \textbf{priority:} how quickly the issue should be resolved, based on the business impact and the prihect imeline.
    \item \textbf{nature:} it refers to technical charateristics of the byg/error (es: functional or security, design or code specific, etc..)
\end{itemize}
Some of these chriteria are objective and measurable while some other are more subective and require some informed evaluation.

\subsection{Why classify bugs?}
The classification can be the first step of investiation for:
\begin {itemize}
    \item conducinf an \textbf{effective} (more targetted) testing acrivity
    \item improving the development process. \\ 
        Some development techniques are more suited for different kinds of task
    \item driving bug-fixing \\
        By knowing the common vulnerabilities of my kind of application I can test those more thoroughly. Like testing SQL injections when we write a webapp.
\end {itemize}

\section{Injection}
An application if vulnerable to attack when user supplied data is not validated, filtered, or sanitized.

\section{SQL injection}
\subsection{Error-based SQL injection}
Un tipo di injection che mira a causare un errore serverside. L'attaccante piò usare l'errore per estrapolare informazioni utili riguardo il sistema.

\subsection{Union-based SQL injection}
Injection che compie delle union SQL per ottenere più dati di quanti sarebbero normalmente accessibili all'utente.

\subsection{Boolean-based Blind SQL injection}
Non si ottiene un messaggio dal server ma si mandano query contenenti operatori booleani per inferire informazioni sulle tabelle.

\subsection{Prevenzione}
Usa i prepared-statement.

\section{Cross-Site Scripting (XSS)}
Attacchi XSS consistono nell'iniezione di codice "maliziono" in siti fidati. \\
Questo è un tipo di attacco \textit{client side}. Quindi l'attaccante ha accesso a dati client side come cookie, token di sessione, etc..
Questo rende l'attacco meno dannoso di una SQL-injection. Ma piò comunque essere molto pericoloso.

\begin{lstlisting}
<?php
    $query = $_GET["query"];
    if (isset($query)) {
        echo "Search results for: ".$query;
    }
?>
\end{lstlisting}
Ci sono molti tag html che possono portare all'esecuzione di codice javascript. Per esempio: script, body, img, ifreame, input, etc..

\paragraph{Possibili soluzioni}
\begin{itemize}
    \item Non fidarsi mai degli input utente
    \item Disattivare i cookie con la flag "htmlonly" se il sito non ne necessita
\end{itemize}

\section{Command Injection}
Command Injection è un tipo di attacco in cui un programma esegue dei comandi. Un utente malintenzionato può manipolare il programma in maniera che i comandi eseguiti siano dannosi o forniscano informazioni. \\
Questo è un tipo di attacco pericoloso in quanto l'esecuzione arbitraria di comandi piò portare a danni ingenti alla macchina o perdita di dati importanti.
\paragraph{Requisiti per l'attacco}
\begin{itemize}
    \item The app shold have privileges/permissions to execute system commands
    \item The app should use user-provided data as part of system commands
    \item The user-procided data should not be escaped/sanitized before use
\end{itemize}

\paragraph{Mitigazione}
\begin{itemize}
    \item Non usare funzioni di esecuzione shell/SO
    \item Non usare input utente nei comandi shell/SO
    \item Sanatizzare l'input
        \begin{itemize}
            \item Whitelist di caratteri/keyword non pericolosi
            \item Blacklist di caratteri/keyword pericolosi
            \item Escaping dell'input
        \end{itemize}
    \item Parapetrizzare i comandi (simile a prepared statements)
    \item Principio di \textit{Least Provilege}
\end{itemize}

Bisogna anche scegliere saggiamente il metodo in cui il programma invia i comandi al sistema operativo. Ad esempio in C c'è il comando \texttt{execvp} che è più sicuro di \texttt{system}. In Java ci sono \texttt{Runtime.exec} e il più sicuro \texttt{ProcessBuilder}.

\section{File Inclusion Attack}
Un attacco di file inclusion mira a programmi che fanno utilizzo di dati salvati su file a runtime. Se l'attaccante può modificare i file (o aggiungerne di nuovi) prima che vengano inclusi può affliggere il programma.

Questo tipo di attacco si suddivide in \textit{Local File Inclusion (LFI)} e \textit{Remote File Inclusion (RFI)}.
Il principale rischio nel primo caso è la trapelazione di informazioni (esponendo file locali all'attaccante), mentre nel secondo caso è più probabile che l'attaccante carichi file dannosi verso il sistema.

\subsection{File access: directory traversal}
L'attaccante usa una serie di "\textit{../}" per esplorare il file system del sistema target. \\
Invece che sanatizzare l'input, questi casi possono essere affrontati configurando permessi adeguati all'applicazione.

\section{Error Handling}
L'error handling è un meccanismo usato per risolvere o gestire errori che si verificano durante l'esecuzione di un pogramma. La gestione degli errori è una parte integrante della sicurezza di un sistema.

Un attaccante inizia dalla fase di ricognizione, in cui deve ottenere informazioni riguardo il sistema che sta attaccando. Queste informazioni possono essere fatte trapelare da messaggi d'errore, stack trace, e altre forme di error handling.

Anche il modo in cui viene implementato il codice piò portare a far trapelare informazioni. Un errore che porta il sistema a crashare può fornire all'attaccante un'intera stack trace dell'errore che include altri dati.

\paragraph{Mitigazione} Vulnerabilità di questo tipo possono essere mitigate filtrando i messaggi di errore che vengono mandati all'utente o gestendo i casi d'errore in modo che non vengano sollevate exception / crash.

\section{Insecure Direct Object Reference (IDOR)}
IDOR is a type of access control vulnerability that arises when an application uses usersupplied input to access objects directly.
A direct object reference occurs when a developer exposes a reference to internal
implementation objects (e.g., files, directories, database keys, session ids, query
parameters) without appropriate validation mechanisms, thus allowing attackers to
manipulate these references to access unauthorized data.

\section{Client-Side Validation}
La validazione di input dal lato del client è utile per alleviare lo stress sul server (evitando un avanti e indietro), ma i controlli vanno sempre replicati sul server alla fine.

\section{Client-Side Filtering}
Certe volte il server manda più informazioni del dovuto al client, fidandosi che sia quest'ultimo a filtrare queste informazioni prima di mostrarle all'utente. Questa è una bad practice per ovvi motivi.

\section{Phases of testing}
\subsection{Verification vs Validation}
\textit{Verification} asks the question "are you building it right?" while \textit{Validation} checks "Are you building the right thing".\\
We do both to ensure that the system does the right thing and that it does so safely

\subsection{Static vs Dynamic Analysis}
L'analisi statica viene svolta sul codice effettivo, mantre quella dinamica è svolta sul processo in esecuzione.

\subsection{Test cases - dynamic}
Un test case descrive una procedura che mette alla prova il sistema
\paragraph{Oracle} "l'oracolo" è l'output atteso del programma nel caso il test case vada a buon fine. Deve essere stabilito manualmente dallo sviluppatore.

\subsection{Test cases - static}
???

\end{document}


