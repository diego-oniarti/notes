\documentclass{article}

\usepackage[utf8]{inputenc}
\usepackage{amsmath}
\usepackage[most]{tcolorbox}
\tcbuselibrary{skins,breakable}

\newtcolorbox{callout}[2][]{breakable,sharp corners, skin=enhancedmiddle jigsaw,parbox=false,
boxrule=0mm,leftrule=2mm,boxsep=0mm,arc=0mm,outer arc=0mm,attach title to upper,
after title={.\ }, coltitle=purple,colback=purple!10,colframe=purple, title={#2},
fonttitle=\bfseries,#1}

\newtcolorbox{warning}[2][]{breakable,sharp corners, skin=enhancedmiddle jigsaw,parbox=false,
boxrule=0mm,leftrule=2mm,boxsep=0mm,arc=0mm,outer arc=0mm,attach title to upper,
after title={!\ }, coltitle=red,colback=red!10,colframe=red, title={#2},
fonttitle=\bfseries,#1}

\newtcolorbox{esempio}[2][]{breakable,sharp corners, skin=enhancedmiddle jigsaw,parbox=false,
boxrule=0mm,leftrule=2mm,boxsep=0mm,arc=0mm,outer arc=0mm,attach title to upper,
after title={:\ }, coltitle=blue,colback=blue!10,colframe=blue, title={#2},
fonttitle=\bfseries,#1}


\usepackage[parfill]{parskip}
\usepackage{multirow}

\title{Distributed Systems}
\author{Diego Oniarti}
\date{Anno 2024-2025}

\begin{document}

\maketitle
\tableofcontents

\newpage
\section{Introduction} 
\paragraph{Prisoner's dilemma}
\begin{center}
\begin{tabular}{r|c c}
    & not confess & confess \\
    \hline
    not confess & 3,3 & 0,4 \\
    confess & 4,0 & 1,1
\end{tabular}
\end{center}
This is a representation of the \textit{prisoner's dilemma} in which the values are not years of sentence but "utilities". We do this because we want a value to maximise instead of minimizing it.

In this classical example, the cell (1,1) where both prisoners confess is an "equilibrium". Once the system settles in that state no agent has an incentive to deviate.

\paragraph{Multiple equilibriums}
Another example could be Alice and Bob deciding where they wanna go on the weekend, with the choices being a concert and a baseball game
\begin{center}
    \begin{tabular}{r | r | c c}
        & & \multicolumn{2}{|c}{bob} \\
        \hline
        & & game & concert \\
        \hline
        \multirow{2}{2em}{alice} & game & 4,1 & 0,0 \\
                                 &  concert &  0,0 & 1, 4\\
    \end{tabular}
\end{center}
In this situation there are two identical equilibriums, so there might be a \textit{convention} to decide between the two of them. In the real world this convention could be "let's be a gentleman and pick what the woman prefers".

\begin{esempio}{Peek at Nash's theorem}
The Nash's theorem states that every game has an equilibrium. This equilibrium could be single, mixed, or other kind of equilibrium.\\
This is a non-constructive theorem, as it gives no indication on how to find the equilibrium.
\end{esempio}

\subsection{Mixed vs Pure equilibriums}


\end{document}
