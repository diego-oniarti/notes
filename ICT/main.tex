\documentclass{article}

\usepackage[utf8]{inputenc}
\usepackage{amsmath}
\usepackage{amssymb}
\usepackage{hyperref}
\usepackage{multirow}
\usepackage[most]{tcolorbox}
\tcbuselibrary{skins,breakable}

\newtcolorbox{callout}[2][]{breakable,sharp corners, skin=enhancedmiddle jigsaw,parbox=false,
boxrule=0mm,leftrule=2mm,boxsep=0mm,arc=0mm,outer arc=0mm,attach title to upper,
after title={.\ }, coltitle=purple,colback=purple!10,colframe=purple, title={#2},
fonttitle=\bfseries,#1}

\newtcolorbox{warning}[2][]{breakable,sharp corners, skin=enhancedmiddle jigsaw,parbox=false,
boxrule=0mm,leftrule=2mm,boxsep=0mm,arc=0mm,outer arc=0mm,attach title to upper,
after title={!\ }, coltitle=red,colback=red!10,colframe=red, title={#2},
fonttitle=\bfseries,#1}

\newtcolorbox{esempio}[2][]{breakable,sharp corners, skin=enhancedmiddle jigsaw,parbox=false,
boxrule=0mm,leftrule=2mm,boxsep=0mm,arc=0mm,outer arc=0mm,attach title to upper,
after title={:\ }, coltitle=blue,colback=blue!10,colframe=blue, title={#2},
fonttitle=\bfseries,#1}

\newcommand{\na}[0]{\ensuremath {\overset{N}{\rightarrow}}}
\newcommand{\rl}[3]{\inference{#1}{#2}\text{ #3}}
\newcommand{\bop}[0]{\ensuremath\oplus}
\newcommand{\appl}[2]{\ensuremath(#1)\ #2}
\newcommand{\st}[3][]{\ensuremath{\displaystyle\frac{#3\hfill}{#2\hfill} \text{#1}}}
\newcommand{\N}{\ensuremath \mathbb N}
\newcommand{\I}{\ensuremath \mathbb I}
\newcommand{\lam}[2]{\ensuremath{\lambda#1.#2}}
\newcommand{\inl}[0]{\ensuremath{\ inl\ }}
\newcommand{\inr}[0]{\ensuremath{\ inr\ }}
\newcommand{\case}[3]{\ensuremath{\text{case}#1\ \text{of}\ \left|\begin{aligned}& #2\\ & #3\end{aligned}\right.}}
\newcommand{\Da}[0]{\ensuremath{\Downarrow}}
\newcommand{\while}[2]{\ensuremath{\text{while }#1\text{ do }#2\text{ end}}}
\newcommand{\for}[3]{\ensuremath{\text{for }i=#1\text{ to }#2\text{ do }#3\text{ end}}}
\newcommand{\mE}[0]{\ensuremath{\mathbb{E}}}
\newcommand{\pair}[1]{\ensuremath{\langle#1\rangle}}
\newcommand{\V}{\ensuremath{\mathcal{V}}}
\newcommand{\cE}{\ensuremath{\mathcal{E}}}
\newcommand{\cD}{\ensuremath{\mathcal{D}}}
\newcommand{\cF}{\ensuremath{\mathcal{F}}}
\newcommand{\IF}[0]{\ensuremath {\text{ if }}}
\newcommand{\THEN}[0]{\ensuremath {\text{ then }}}
\newcommand{\ELSE}[0]{\ensuremath {\text{ else }}}
\newcommand{\AND}[0]{\ensuremath {\text{ and }}}
\newcommand{\OR}[0]{\ensuremath {\text{ or }}}
\newcommand{\unpack}[3]{\ensuremath{\text{unpack } #1 \text{ as }\langle #2 \rangle\text{ in }#3}}
\newcommand{\pack}[2]{\ensuremath{\text{pack } \pair{#1} \text{ as } #2 }}
\newcommand{\te}[1]{\text{#1}}
\newcommand{\ls}[0]{\ensuremath{\leadsto^{*}}}
\newcommand{\LET}[0]{\ensuremath{\text{ let }}}
\newcommand{\TIN}[0]{\ensuremath{\text{ in }}}
\newcommand{\NEW}[0]{\ensuremath{\text{ new }}}


\title{Appunti ICT}
\author{Diego Oniarti}
\date{Anno 2024-2025}

\begin{document}

\maketitle
\tableofcontents

\section{Note sul corso}
Professore: Carlo Pasquini.

\section{Innovation}
\subsection{Novelty vs Innovation}
Innovation is the \textbf{implementation} of a new or significantly improved product, or process, a new marketing method, or a new organizational method in business practise, workplace organizaiton or external relations. \\
Innovation = invention + exploitation \\
An innovation:
\begin{itemize}
    \item there is a market need
    \item has a strong value proposition
    \item solves a problem
    \item does the job better than other
    \item there are custoners willing to pay
    \item it is technologically feasable
    \item it is financially sustainable
\end{itemize}

\subsection{Why do we need innovation?}
Bisinesses \textbf{must} innovate to keep up with the competition. The benefit to the customer is a byproduct of the race between the companies. \\
For the business, innovation is a necessity to survive
\begin{itemize}
    \item to survive competition in the short and long term
    \item to acquire market share
    \item to look for premium price, better margins, higher added value
    \item innovation is the \textbf{real} protection of your IPRs
\end{itemize}

\subsection{Kinds of innovation}
Innovation can be a \textbf{process}, a \textbf{prduct}, or an entire \textbf{business model}. 
The only thing that matters is that it is \textbf{NEW}.

\subsection{Radical vs Incrementa innovation}
\paragraph{Incremental}
We call an innovation "\textit{incremental}" when it improves on something that already exists. \\
It lives in a short cycle, with recurrent improvements on an existing solution.

It's a reliable investment for the companies because it relies on mainstream customers in major markets.

\paragraph{Radical}
Radical innovation is the departure from existing tecnologies and solutions to offer something \textbf{entirely} new.
Radical innovations are those innovations that bring to a market a very different value proposition that hadn't been aviable previously,

\subsection{Theory of disruptive innovation}
In his book "disruptive innovations", Christenses presents "the innovator's \\ dilemma". This dilemma is the one between investing in a possibly risky innovation and playing it safe.

One of the risks in NOT investing in a disruptive technology is that of a competitor investing in it and stealing potential market shares.

\section{Types of problems}
\begin{itemize}
    \item \textbf{Latent problems:} Customers have a problem but the don't know it. \\
        Esempi: \begin{itemize}
            \item Spotify / music streaming
            \item Smart voice assistants
        \end{itemize}
    \item \textbf{Passive problems:} Customers know they have a problem but are not motivated or aware of the opportunity to change
    \item \textbf{Urgent problems:} Customers recognize a problem and are searching for a solution
\end{itemize}

\section{My Idea}
\begin{itemize}
    \item Who: Music enthusiasts with short attention spans
    \item Problem (a latent one): Someone suggests a song to you but you forget the title, or to listen to it
    \item Solution: An app where your friends can leave you song suggestions, and you see them as a playlist
\end{itemize}

\section{2024-09-30}
Idee:
\begin{itemize}
    \item Data protection with AI - Donathan - meh
    \item Toilet As A Service - Fabio M. - meme
    \item Picture2Video - Marco L. - not feasable? cool
    \item University Summary - Silvanus - silvanus
    \item Let's play - Mario S. - no
    \item Automated Garden - Hunter B. - carino
    \item Datapool -  Riccardo S. - carino
\end{itemize}

\begin{tabular}{r|l l l|}
      & Nome progetto & presentatore & opinione \\ \hline
    1& Data protection with AI & Donathan & meh  \\ \hline
    2& Toilet As A Service & Fabio M. & meme \\ \hline
    3& Picture2Video & Marco L. & not feasable? cool \\ \hline
    4& University Summary & Silvanus & silvanus \\ \hline
    5& Let's play & Mario S. & no \\ \hline
    6& Automated Garden & Hunter B. & carino \\ \hline
    7& Datapool &  Riccardo S. & carino \\ \hline
    8& Lookify & Sano & bah \\ \hline
    9& A healthy app & Giacomo S. & fattibile \\ \hline
    10& Deliver your groceries & Maria A. & Glovo \\ \hline
    11& Ski App & Ludovico & unfeasable? \\ \hline
    12& OWD & Tommaso S. & Ci sta \\ \hline
    13& Chatbot Monitor & Lerenza B. & mercato è limitato \\ \hline
    14& Bike Aware & Sebastian W. & buono. Ma pratico? \\ \hline
    15& Tidy Storage & alessio z. & lazy \\ \hline
    16& SmartPack & dems c. & esiste \\ \hline
    17& DevConnect & Davide Z. & Github \\ \hline
    18& OpenSourceSec& Simone A. & Unfeasable \\ \hline
    19& Where I Train & Massimiliano & Good \\ \hline
    20& Just Do It & Leonardo & no \\ \hline
    21& Garbage Collector & Ale & Anche solo per il nome \\ \hline
    22& Fridge Management & Alberto M. & Tedioso \\ \hline
    23& Pong Stats & Francesco D. & Carino ma AI \\ \hline
    24& Magic Mirror & Davide M & Good \\ \hline
    25& Vending Machine Finder & gabriele M. & good \\ \hline
    26& Delivery food 4 students &  & Deliveru (2) \\ \hline
    27& Accident report ai & saras & Difficile ma bello \\ \hline
    28& Artva & & No \\ \hline
    29& SW4G & Samuel & c'è già OWD \\ \hline
    30& Gamification & & carino \\ \hline
    31& Smart Stoves & Gabriele V. & esiste, ma carino \\ \hline
    32& Traffic avoidance & Francesco S. & unfeasable \\ \hline
    33& Opportunity Seeker & Quentin M. & non ascoltato \\ \hline
    34& One Track & Enrico T. & small market \\ \hline
    35& Healthcare at home & Salvatore C. & too hard \\ \hline
    36& SOS artists & Denise C. & gut \\ \hline
    37& Slow Bonding & Matteo D. & bah \\ \hline
    38& Smart Shoes & Nicola B. & carino \\ \hline
    39& NoWasteFood & Andrea V. & too good to go \\ \hline
\end{tabular}

\section{Lean}
In the lean methodology the assumpion is that its possible to figure out the \textbf{unknowns} of a business in advance, before you raise money and actually execute the idea.

\subsection{Principle 1}
Rather than engaging in months of planning and research, entrepreneurs/intrapreneurs accept that all they have on day one is a series of untested hypotheses.

\paragraph{How do we test these hypothesis?}
\begin{enumerate}
    \item State your hypothesis \\
        L'ipotesi deve essere \textbf{specifica} e avere uno scope limitato
    \item Design experiment \\
        L'esperimento deve essere semplice e di tipo binario (pass/fail)
    \item Test \\
        Il test deve involgere l'utente
    \item Insignt
\end{enumerate}

\subsection{Principle 2 - Agile Development}
\paragraph{Minimum Viable Product}
A \textit{MVP} is a version of a product with just enough dfeatures to be useavle by early customers\footnote{\textit{Early Adopters} are those wha are actively looking for a solution. This makes them more forgiving and more likely to give feedback} who can them provide feedback for forther development. 
A minimum viable product must: \begin{enumerate}
    \item Test a product hypothesis with minumal resources
    \item Accelerate learnings
    \item Reduce wasted engineers hours
    \item Get the product to early customers as soon as possible
\end{enumerate}

\begin{esempio}{es}
    \paragraph{Problem}
    Imagine you have validated that office workers are in need of healthy food for their lunch break and the solution you have in mind is a ghost kitchen where to prepare food to be ordered though an app.
    \paragraph{Possible Solution}
    A possible MVP is the food dropshipping. Where we don't open a kitchen yet but rely on another service like deliveroo to do it for us. 
\end{esempio}

It is acceptable in the development of the MVPs to lose some money.

\begin{enumerate}
    \item Sketch
    \item Wireframe
    \item Mockup
    \item Video
    \item ...
\end{enumerate}

\begin{warning}{Never hire a banana}
\end{warning}

\section{Business Modelling}
A business model describes the rationale of how an organization creates, delivers, and captures value
\subsection{Razor and Blade business model}

What constitutes a business model?
\begin{enumerate}
    \item Value proposition
    \item Customer Segments
    \item Distribution Chennels
    \item Custoner Relationships
    \item Revenue streams
    \item Resources
    \item Activities
    \item Partners
    \item Cost structure
\end{enumerate}

\subsection{Value Proposition}
\subsection{Customer Segments}
\begin{itemize}
    \item mass market
    \item niche market
    \item diversified market
    \item ...
\end{itemize}

\subsection{Channels}
How a company reaches its customer segments.
Channels are vistomers touch points that play an importnt role in the customer experience.

They divide in \textit{direct} and \textit{indirect} channels.
\paragraph{Direct} higher margins but costly
\paragraph{Indirect} lower margins but cheaper

\subsection{Customer Relationships}
The type of telationships a company enstablishes with specific customer sefments. Customer relationships have impact on customer acquisition and retention.

\begin{itemize}
    \item GET: acquire customers \\
        website, app store, search, emails, blogs, free triels, etc..
    \item KEEP: Interact, retain \\
        Customization, blogs, online hels, product tips, affiliates
    \item GROW: New revenue, referrals \\
        upsell, cross sell, upgrades, reorders, refer friends
\end{itemize}

\end{document}


