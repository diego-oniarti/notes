\documentclass{article}

\usepackage[utf8]{inputenc}
\usepackage{amsmath}
\usepackage{amssymb}
\usepackage{hyperref}
\usepackage{multirow}
\usepackage[most]{tcolorbox}
\tcbuselibrary{skins,breakable}

\newtcolorbox{callout}[2][]{breakable,sharp corners, skin=enhancedmiddle jigsaw,parbox=false,
boxrule=0mm,leftrule=2mm,boxsep=0mm,arc=0mm,outer arc=0mm,attach title to upper,
after title={.\ }, coltitle=purple,colback=purple!10,colframe=purple, title={#2},
fonttitle=\bfseries,#1}

\newtcolorbox{warning}[2][]{breakable,sharp corners, skin=enhancedmiddle jigsaw,parbox=false,
boxrule=0mm,leftrule=2mm,boxsep=0mm,arc=0mm,outer arc=0mm,attach title to upper,
after title={!\ }, coltitle=red,colback=red!10,colframe=red, title={#2},
fonttitle=\bfseries,#1}

\newtcolorbox{esempio}[2][]{breakable,sharp corners, skin=enhancedmiddle jigsaw,parbox=false,
boxrule=0mm,leftrule=2mm,boxsep=0mm,arc=0mm,outer arc=0mm,attach title to upper,
after title={:\ }, coltitle=blue,colback=blue!10,colframe=blue, title={#2},
fonttitle=\bfseries,#1}

\newcommand{\na}[0]{\ensuremath {\overset{N}{\rightarrow}}}
\newcommand{\rl}[3]{\inference{#1}{#2}\text{ #3}}
\newcommand{\bop}[0]{\ensuremath\oplus}
\newcommand{\appl}[2]{\ensuremath(#1)\ #2}
\newcommand{\st}[3][]{\ensuremath{\displaystyle\frac{#3\hfill}{#2\hfill} \text{#1}}}
\newcommand{\N}{\ensuremath \mathbb N}
\newcommand{\I}{\ensuremath \mathbb I}
\newcommand{\lam}[2]{\ensuremath{\lambda#1.#2}}
\newcommand{\inl}[0]{\ensuremath{\ inl\ }}
\newcommand{\inr}[0]{\ensuremath{\ inr\ }}
\newcommand{\case}[3]{\ensuremath{\text{case}#1\ \text{of}\ \left|\begin{aligned}& #2\\ & #3\end{aligned}\right.}}
\newcommand{\Da}[0]{\ensuremath{\Downarrow}}
\newcommand{\while}[2]{\ensuremath{\text{while }#1\text{ do }#2\text{ end}}}
\newcommand{\for}[3]{\ensuremath{\text{for }i=#1\text{ to }#2\text{ do }#3\text{ end}}}
\newcommand{\mE}[0]{\ensuremath{\mathbb{E}}}
\newcommand{\pair}[1]{\ensuremath{\langle#1\rangle}}
\newcommand{\V}{\ensuremath{\mathcal{V}}}
\newcommand{\cE}{\ensuremath{\mathcal{E}}}
\newcommand{\cD}{\ensuremath{\mathcal{D}}}
\newcommand{\cF}{\ensuremath{\mathcal{F}}}
\newcommand{\IF}[0]{\ensuremath {\text{ if }}}
\newcommand{\THEN}[0]{\ensuremath {\text{ then }}}
\newcommand{\ELSE}[0]{\ensuremath {\text{ else }}}
\newcommand{\AND}[0]{\ensuremath {\text{ and }}}
\newcommand{\OR}[0]{\ensuremath {\text{ or }}}
\newcommand{\unpack}[3]{\ensuremath{\text{unpack } #1 \text{ as }\langle #2 \rangle\text{ in }#3}}
\newcommand{\pack}[2]{\ensuremath{\text{pack } \pair{#1} \text{ as } #2 }}
\newcommand{\te}[1]{\text{#1}}
\newcommand{\ls}[0]{\ensuremath{\leadsto^{*}}}
\newcommand{\LET}[0]{\ensuremath{\text{ let }}}
\newcommand{\TIN}[0]{\ensuremath{\text{ in }}}
\newcommand{\NEW}[0]{\ensuremath{\text{ new }}}


\usepackage{algorithm2e}
\newcommand\mycommfont[1]{\footnotesize\ttfamily\textcolor{blue}{#1}}
\SetCommentSty{mycommfont}

\RestyleAlgo{ruled}

\title{Minesweeper solver}
\author{Diego Oniarti}
\date{}

\begin{document}

\maketitle
% \tableofcontents

\section{Complessità di una mossa}
\paragraph {Metodo naive}
\begin{gather*}
    O\left( \binom{n^2}{m} \cdot n^2 \right) \\
    n^2 = \#celle  \\
    m = \#bombe \\
\end{gather*}

\paragraph {Permutazioni simili}
\begin{gather*}
    O\left( \binom{n^2}{m} \right) \\
    n^2 = \#celle  \\ 
    m = \#bombe \\
\end{gather*}

\newpage

\section{Algoritmi di risoluzione}
L'idea è di implementare diversi algoritmi di risoluzione e di provarli tutti in ordine di complessità crescente. Se uno degli algoritmi riesce a far progredire la partita inizia il processo di nuovo sulla board ottenuta.

\subsection{Trivial cells}
Alcune celle sono trivialmente libere o bombe. Questo nel caso il numero di una cella sia il numero di flag o celle libere circostanti
\begin{algorithm}
    \caption{Trivial cells}
    \label{algo:trivial}
    \ForAll{$c\in celle$} {
        \If {c.hidden} {
            continue\;
        }
        $n \gets c.num()$\;
        $neigh \gets c.neighbors().filter(hidden)$\;
        \If {$n = |neigh|$} {
            \ForAll {$f\in neigh$} {
                $f.flag()$\;
            }
            continue\;
        }
        \If {$n = |neigh.filter(flagged)|$} {
            \ForAll {$f\in neigh.filter(!flagged)$} {
                $f.click()$\tcp*{Counts as progress}
            }
        }
    }
\end{algorithm}

\newpage
\subsection{Set theory}
L'implementazione di questo algoritmo assume che sia già stato svolto \textit{Algo.\ref{algo:trivial}}

\newpage
\subsection{Brute-force}
Questo algoritmo è molto pesante, quindi viene usato solo se $\binom{celle\ libere}{bombe}$ si tiene sotto un certo limite
\begin{algorithm}
    \caption{Brute-force method}
    $p\_count \gets 0$\;
    \ForAll {$p\in Permutations$}{
        \If {!is\_possible($p$)} { 
            continue\; 
        }
        $c\_count++$\;
        \ForAll {$b\in p$} {
            $b.count++$\;
        }
    }
    $progress \gets false$\;
    \ForAll{$c\in celle$} {
        $p \gets \frac{c.count}{p\_count}$\;
        \If {$p==1$} {
            $p.flag()$ \tcp*{Reduce number of bombs}
            $progress \gets true$\;
        }
        \If {$p==0$} {
            $p.click()$ \tcp*{Expand neighbors if possible}
            $progress \gets true$\;
        }
    }
\end{algorithm}

\end{document}
