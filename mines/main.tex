\documentclass{article}

\usepackage[utf8]{inputenc}
\usepackage{amsmath}
\usepackage[most]{tcolorbox}
\tcbuselibrary{skins,breakable}

\newtcolorbox{callout}[2][]{breakable,sharp corners, skin=enhancedmiddle jigsaw,parbox=false,
boxrule=0mm,leftrule=2mm,boxsep=0mm,arc=0mm,outer arc=0mm,attach title to upper,
after title={.\ }, coltitle=purple,colback=purple!10,colframe=purple, title={#2},
fonttitle=\bfseries,#1}

\newtcolorbox{warning}[2][]{breakable,sharp corners, skin=enhancedmiddle jigsaw,parbox=false,
boxrule=0mm,leftrule=2mm,boxsep=0mm,arc=0mm,outer arc=0mm,attach title to upper,
after title={!\ }, coltitle=red,colback=red!10,colframe=red, title={#2},
fonttitle=\bfseries,#1}

\newtcolorbox{esempio}[2][]{breakable,sharp corners, skin=enhancedmiddle jigsaw,parbox=false,
boxrule=0mm,leftrule=2mm,boxsep=0mm,arc=0mm,outer arc=0mm,attach title to upper,
after title={:\ }, coltitle=blue,colback=blue!10,colframe=blue, title={#2},
fonttitle=\bfseries,#1}



\usepackage{algorithm2e}
\newcommand\mycommfont[1]{\footnotesize\ttfamily\textcolor{blue}{#1}}
\SetCommentSty{mycommfont}

\RestyleAlgo{ruled}

\title{Minesweeper solver}
\author{Diego Oniarti}
\date{}

\begin{document}

\maketitle
% \tableofcontents

\section{Complessità di una mossa}
\paragraph {Metodo naive}
\begin{gather*}
    O\left( \binom{n^2}{m} \cdot n^2 \right) \\
    n^2 = \#celle  \\
    m = \#bombe \\
\end{gather*}

\paragraph {Permutazioni simili}
\begin{gather*}
    O\left( \binom{n^2}{m} \right) \\
    n^2 = \#celle  \\ 
    m = \#bombe \\
\end{gather*}

\newpage

\section{Algoritmi di risoluzione}
L'idea è di implementare diversi algoritmi di risoluzione e di provarli tutti in ordine di complessità crescente. Se uno degli algoritmi riesce a far progredire la partita inizia il processo di nuovo sulla board ottenuta.

\subsection{Trivial cells}
Alcune celle sono trivialmente libere o bombe. Questo nel caso il numero di una cella sia il numero di flag o celle libere circostanti
\begin{algorithm}
    \caption{Trivial cells}
    \label{algo:trivial}
    \ForAll{$c\in celle$} {
        \If {c.hidden} {
            continue\;
        }
        $n \gets c.num()$\;
        $neigh \gets c.neighbors().filter(hidden)$\;
        \If {$n = |neigh|$} {
            \ForAll {$f\in neigh$} {
                $f.flag()$\;
            }
            continue\;
        }
        \If {$n = |neigh.filter(flagged)|$} {
            \ForAll {$f\in neigh.filter(!flagged)$} {
                $f.click()$\tcp*{Counts as progress}
            }
        }
    }
\end{algorithm}

\newpage
\subsection{Set theory}
L'implementazione di questo algoritmo assume che sia già stato svolto \textit{Algo.\ref{algo:trivial}}

\newpage
\subsection{Brute-force}
Questo algoritmo è molto pesante, quindi viene usato solo se $\binom{celle\ libere}{bombe}$ si tiene sotto un certo limite
\begin{algorithm}
    \caption{Brute-force method}
    $p\_count \gets 0$\;
    \ForAll {$p\in Permutations$}{
        \If {!is\_possible($p$)} { 
            continue\; 
        }
        $c\_count++$\;
        \ForAll {$b\in p$} {
            $b.count++$\;
        }
    }
    $progress \gets false$\;
    \ForAll{$c\in celle$} {
        $p \gets \frac{c.count}{p\_count}$\;
        \If {$p==1$} {
            $p.flag()$ \tcp*{Reduce number of bombs}
            $progress \gets true$\;
        }
        \If {$p==0$} {
            $p.click()$ \tcp*{Expand neighbors if possible}
            $progress \gets true$\;
        }
    }
\end{algorithm}

\end{document}
