\documentclass{article}
\usepackage[top=2cm,bottom=2cm,left=2cm,right=3.5cm]{geometry}
\usepackage[utf8]{inputenc}
\usepackage[most]{tcolorbox}
\tcbuselibrary{skins,breakable}

\newtcolorbox{callout}[2][]{breakable,sharp corners, skin=enhancedmiddle jigsaw,parbox=false,
boxrule=0mm,leftrule=2mm,boxsep=0mm,arc=0mm,outer arc=0mm,attach title to upper,
after title={.\ }, coltitle=purple,colback=purple!10,colframe=purple, title={#2},
fonttitle=\bfseries,#1}

\newtcolorbox{warning}[2][]{breakable,sharp corners, skin=enhancedmiddle jigsaw,parbox=false,
boxrule=0mm,leftrule=2mm,boxsep=0mm,arc=0mm,outer arc=0mm,attach title to upper,
after title={!\ }, coltitle=red,colback=red!10,colframe=red, title={#2},
fonttitle=\bfseries,#1}

\newtcolorbox{esempio}[2][]{breakable,sharp corners, skin=enhancedmiddle jigsaw,parbox=false,
boxrule=0mm,leftrule=2mm,boxsep=0mm,arc=0mm,outer arc=0mm,attach title to upper,
after title={:\ }, coltitle=blue,colback=blue!10,colframe=blue, title={#2},
fonttitle=\bfseries,#1}


\usepackage[parfill]{parskip}

\usepackage{graphicx}
\usepackage{algorithm2e}

\title{Advanced Programming for Cryptographic Methods}
\author{Diego Oniarti}
\date{Anno 2025-2026}

\begin{document}

\maketitle
\tableofcontents

\section{introduction}
\paragraph{Schedule}
Iniziamo alle 14:00

\paragraph{topics}
\begin{itemize}
    \item introduction to applied cryptography. Not the quantum stuff yet.
    \item How the theoretical aspects of cryptography translate on physical machines
    \item Software development and security
    \item JAVA using JCA and JCR
    \item C and OpenSSL
\end{itemize}

\paragraph{exam}
\begin{itemize}
    \item exercises assigned burring the semester (0 to 5 points)
    \item software project and report (0 to 27 points)
    \item oral exam with questions stating from the project and spanning the whole course (-5 to 5 points)
\end{itemize}
no fixed groups for exercises

\paragraph{Cryptography overview}
Two actors (Alice and Bob) have to communicate over an \textit{insecure} channel. The channel is not secure because a third actor (Eve) is trying to tamper with the communication.

Eve could have different capabilities in terms of computation power, system knowledge, etc.\\
Existing systems have to constantly be updated to keep up with Eve's potential evolutions.

Three things that we want from our systems are
\begin{itemize}
    \item Confidentiality
    \item Integrity
    \item Availability
\end{itemize}
Cryptography can work on the first two. But availability is not our problem.

\paragraph{Theoretical vs applied}
In theory numbers, keys, and other models can be arbitrarily big. In the real world whey need to be stored into memory or storage and processed in binary.

\paragraph{Hash Function}
Function $f(x)=d$ that takes an element $x$ from a set and returns a sequence of bits $d$.
\begin{itemize}
    \item $d$ must be of fixed size
    \item $f$ must not be reversible
    \item $f(x)$ must be unique. Collisions violate \textit{integrity}.\\
        In the real word collisions are inevitable. But we must strive to minimize them as much as possible
\end{itemize}

\paragraph{Cryptosystem}
A cryptosystem is a 5-tuple $<E,D,M,K,C>$ where
\begin{itemize}
    \item $E$ is an encryption algorithm
    \item $D$ is a decryption algorithm
    \item $M$ is the set of plaintexts
    \item $K$ is the set of keys
    \item $X$ is the set of ciphertextx
\end{itemize}

$E$ and $D$ can be characterized as functions

\begin{align*}
    E:M\times K   & \to C \\
    D: X \times K & \to M \\
    D(E(m,k),k)   & = m
\end{align*}

\paragraph{Kerckhoff's principle}
A cryptosystem should be secure even if everything about the system, except the key, is public knowledge

This principle makes interoperability of cryptographic primitives possible.

\paragraph{modularity}
Cryptographic primitives are not \textit{modular}. This means that even a combination of secure primitives can create an insecure application.

\section{Hash Functions}
Hash functions should be strictly one-way and it should be unfeasible to generate a collision on demand.

We will use the \textit{Merkle construction} for this.\\
The input message is partitioned into $t$ number of bit blocks, each one being $n$ bits.\\
If necessary the final block is padded so that it is of the same length as the others.

The construction is parametric with respect to a compression function $f$ which takes $2$ arguments.

We take an initialization vector with the size $m$ of the final message. Then $f$ is applied iteratively on the initialization vector, passing a different block of the message as the second parameter each time.

The compression function in \texttt{SHA256} takes in some constant as well to further increase the difficulty in cracking it. It is a composition of linear and non-linear operations, like bit-shifts and xor operations.

\begin{figure}
    \center
    \includegraphics[width=0.8\linewidth]{images/sha.png}
    \caption{Compression function in SHA256}
\end{figure}

\subsection{Linear functions, diffusion and digression}

\end{document}
