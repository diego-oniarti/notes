\documentclass{article}

\usepackage[utf8]{inputenc}
\usepackage{amsmath}
\usepackage[most]{tcolorbox}
\tcbuselibrary{skins,breakable}

\newtcolorbox{callout}[2][]{breakable,sharp corners, skin=enhancedmiddle jigsaw,parbox=false,
boxrule=0mm,leftrule=2mm,boxsep=0mm,arc=0mm,outer arc=0mm,attach title to upper,
after title={.\ }, coltitle=purple,colback=purple!10,colframe=purple, title={#2},
fonttitle=\bfseries,#1}

\newtcolorbox{warning}[2][]{breakable,sharp corners, skin=enhancedmiddle jigsaw,parbox=false,
boxrule=0mm,leftrule=2mm,boxsep=0mm,arc=0mm,outer arc=0mm,attach title to upper,
after title={!\ }, coltitle=red,colback=red!10,colframe=red, title={#2},
fonttitle=\bfseries,#1}

\newtcolorbox{esempio}[2][]{breakable,sharp corners, skin=enhancedmiddle jigsaw,parbox=false,
boxrule=0mm,leftrule=2mm,boxsep=0mm,arc=0mm,outer arc=0mm,attach title to upper,
after title={:\ }, coltitle=blue,colback=blue!10,colframe=blue, title={#2},
fonttitle=\bfseries,#1}


\usepackage[parfill]{parskip}
\usepackage{hyperref}


\title{Automated Reasoning and Formal Verification}
\author{Diego Oniarti}
\date{Anno 2024-2025}

\begin{document}

\maketitle
\tableofcontents

\section{25-02-2025}
\subsection*{intro}
Slides will be on his \href{https://disi.unitn.it/rseba/DIDATTICA/arfv2025/}{webpage} along with the recordings.

The exam will consist of a script and an oral exam on the topics of the whole course.

\subsection*{boolean/propositional logic}
A propositional \textbf{formula} can be:
\begin{itemize}
    \item $\top$, $\bot$
    \item Propositional \textbf{atoms} $A_1, A_2, \dots, A_n$
    \item A combination of other formulas. If $\phi_1$ and $\phi_2$ are formulas, so are:
        \begin{itemize}
            \item $\neg \phi_1$
            \item $\phi_1\wedge\phi_2$
            \item $\phi_1\vee\phi_2$
            \item $\phi_1\to\phi_2$
            \item $\phi_1\leftarrow\phi_2$
            \item $\phi_1\leftrightarrow\phi_2$
            \item $\phi_1\oplus\phi_2$
        \end{itemize}
\end{itemize}

We define a function $Atoms(\phi)$ representing the set $\{A_1,\dots,A_n\}$ of atoms in $\phi$

A \textbf{clause} is a disjunction of literals $\bigvee_j l_j$ or $(A_1\vee \neg A_2 \vee ...)$

A \textbf{cube} is a conjunction of literals $\bigwedge_j l_j$ or $(A_1\wedge \neg A_2 \wedge ...)$

\subsection*{trees and DAGS}
A tree is a natural representation of a 

\end{document}
