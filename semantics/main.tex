\documentclass{article}

\usepackage[utf8]{inputenc}
\usepackage{amsmath}

\usepackage[inference]{semantic}
\usepackage{amsmath} \allowdisplaybreaks % lets align equations break over pages.
\usepackage{hyperref}
\usepackage[amsmath,hyperref,amsthm]{ntheorem}
\usepackage{thmtools}
\usepackage[capitalize]{cleveref}
\usepackage{mathpartir}
\usepackage{stmaryrd}
\usepackage{amssymb}
\usepackage{latexsym}
\usepackage{color}
\usepackage[dvipsnames]{xcolor} % the best way to colour text
\usepackage[colorinlistoftodos]{todonotes} 
\usepackage{tikz} \usetikzlibrary{positioning,shadows,arrows,calc,backgrounds,fit,shapes,snakes,shapes.multipart,decorations.pathreplacing,shapes.misc,patterns}
\usepackage{xspace}
\usepackage{scalerel}
\usepackage{bm} %bold math. a mess to use.
\usepackage{bussproofs} % for logic-style proofs
\usepackage{xfrac}

\hypersetup{ pdfpagemode=UseOutlines, colorlinks=true, linkcolor=red, citecolor=blue }

%%%%%%%%%%%%%%%%%%%%%%%%%%%%%%%%%%%%%%%%%%%%%%%%%%%%%%%%%%%%%%%%%%%%%%%%%%%%%%%%%%%%%%%%%%%%%%%%%%%%%%%%%%%%%%%%%%%%%%%%%%%%%%%%%%%%%%%%%%%%%%%%%%%%%%%%%%%%%%%%%%%%%%%%%%%%%%%%%
%%%%%%%%%%%%%%%%%%%%%%%%%%%%%%%%%%%%%%%%%%%%%%%%%%%%%%%%%%%							COMMANDS						  %%%%%%%%%%%%%%%%%%%%%%%%%%%%%%%%%%%%%%%%%%%%%%%%%%%%%%%%%%%
%%%%%%%%%%%%%%%%%%%%%%%%%%%%%%%%%%%%%%%%%%%%%%%%%%%%%%%%%%%%%%%%%%%%%%%%%%%%%%%%%%%%%%%%%%%%%%%%%%%%%%%%%%%%%%%%%%%%%%%%%%%%%%%%%%%%%%%%%%%%%%%%%%%%%%%%%%%%%%%%%%%%%%%%%%%%%%%%%

%%%%%%%%%%%%%%%%%%%%%%%%%%%%%%%%%%%%%%%%%%%%%%%%%%%%%%%%%%%
%	TODO annotations
%%%%%%%%%%%%%%%%%%%%%%%%%%%%%%%%%%%%%%%%%%%%%%%%%%%%%%%%%%%
\newcommand{\MP}[1]{\todo[color=blue!30]{MP TODO: #1}}
\newcommand{\MPin}[1]{\todo[color=blue!30,inline]{MP TODO: #1}}

%%%%%%%%%%%%%%%%%%%%%%%%%%%%%%%%%%%%%%%%%%%%%%%%%%%%%%%%%%%
%	Math formatting
%%%%%%%%%%%%%%%%%%%%%%%%%%%%%%%%%%%%%%%%%%%%%%%%%%%%%%%%%%%
% some shortcuts
\newcommand{\mi}[1]{\ensuremath{\mathit{#1}}}
\newcommand{\mr}[1]{\ensuremath{\mathrm{#1}}}
\newcommand{\mt}[1]{\ensuremath{\texttt{#1}}}
\newcommand{\mtt}[1]{\ensuremath{\mathtt{#1}}}
\newcommand{\mf}[1]{\ensuremath{\mathbf{#1}}}
\newcommand{\mk}[1]{\ensuremath{\mathfrak{#1}}}
\newcommand{\mc}[1]{\ensuremath{\mathcal{#1}}}
\newcommand{\ms}[1]{\ensuremath{\mathsf{#1}}}
\newcommand{\mb}[1]{\ensuremath{\mathbb{#1}}}
\newcommand{\msc}[1]{\ensuremath{\mathscr{#1}}}

\DeclareMathOperator\mydefsym{\ensuremath{\iangleq}}
\newcommand{\bnfdef}[0]{\ensuremath{\mathrel{::=}}}
\newcommand{\isdef}[0]{\ensuremath{\mathrel{\overset{\makebox[0pt]{\mbox{\normalfont\tiny\sffamily def}}}{=}}}}

% http://tex.stackexchange.com/questions/5502/how-to-get-a-mid-binary-relation-that-grows
\newcommand{\relmiddle}[1]{\mathrel{}\middle#1\mathrel{}}
\newcommand{\myset}[2]{\ensuremath{\left\{#1 ~\relmiddle|~ #2\right\}}}

\newcommand{\divr}[0]{\ensuremath{\!\!\Uparrow}\xspace}
\newcommand{\term}[0]{\ensuremath{\!\!\Downarrow}\xspace}

%%%%%%%%%%%%%%%%%%%%%%%%%%%%%%%%%%%%%%%%%%%%%%%%%%%%%%%%%%%
%	Compiler 
%%%%%%%%%%%%%%%%%%%%%%%%%%%%%%%%%%%%%%%%%%%%%%%%%%%%%%%%%%%
\newcommand{\genlang}[2]{\ensuremath{\lambda^{#1}_{#2}}}
\newcommand{\genlangF}[2]{\ensuremath{F^{#1}_{#2}}}

\newcommand{\ulc}[0]{\bl{\genlang{u}{}}}
\newcommand{\stlc}[0]{\src{\genlang{\tau}{}}}
\newcommand{\sysf}[0]{\oth{\genlangF{\forall}{}}}

\newcommand{\funname}[1]{\mtt{#1}}
\newcommand{\fun}[2]{\ensuremath{{\bl{\funname{#1}\left(#2\right)}}}\xspace}
\newcommand{\dom}[1]{\fun{dom}{#1}}

%%%%%%%%%%%%%%%%%%%%%%%%%%%%%%%%%%%%%%%%%%%%%%%%%%%%%%%%%%%
%	Language shortcuts
%%%%%%%%%%%%%%%%%%%%%%%%%%%%%%%%%%%%%%%%%%%%%%%%%%%%%%%%%%%
\newcommand{\ctx}[0]{\ensuremath{\mk{C}}}
\newcommand{\ctxh}[1]{\ctx\hole{#1}}
\newcommand{\hole}[1]{\ensuremath{\left[#1\right]}}
\newcommand{\evalctx}[0]{\ensuremath{\mb{E}}}

\newcommand{\srce}[0]{\src{\emptyset}\xspace}
\newcommand{\trge}[0]{\trgb{\emptyset}\xspace}
\newcommand{\come}[0]{\com{\emptyset}\xspace}
\newcommand{\othe}[0]{\oth{\emptyset}\xspace}

%%%%%%%%%%%%%%%%%%%%%%%%%%%%%%%%%%%%%%%%%%%%%%%%%%%%%%%%%%%
%	Language formatting
%%%%%%%%%%%%%%%%%%%%%%%%%%%%%%%%%%%%%%%%%%%%%%%%%%%%%%%%%%%
\newcommand{\neutcol}[0]{black}
\newcommand{\stlccol}[0]{RoyalBlue}
\newcommand{\ulccol}[0]{RedOrange}
\newcommand{\commoncol}[0]{black}    % CarnationPink
\newcommand{\othercol}[0]{CarnationPink}

\newcommand{\col}[2]{\ensuremath{{\color{#1}{#2}}}}

\newcommand{\src}[1]{\ms{\col{\stlccol}{#1}}}
\newcommand{\trgb}[1]{\ensuremath{\bm{\col{\ulccol }{#1}}}}
\newcommand{\trg}[1]{{\mf{\col{\ulccol }{#1}}}}
\newcommand{\oth}[1]{\mi{\col{\othercol }{#1}}}
% MARCO: \bm is notorious to break things around. it's there only to make bold math letters. we can remove it if necessary.
% it is currently removed -- the paretheses are still there though -- as it did go beyond its scope, i did not know how to remove it (\mr did nont work)
% it was affecting stuff inside the compilation brackets, making source stuff bold ... 
%if we know of a solution, we can add \bm at the beginning here and the bold-removal command in the core of \compgen
\newcommand{\bl}[1]{\col{\neutcol }{#1}}
\newcommand{\com}[1]{\mi{\col{\commoncol }{#1}}}

%%%%%%%%%%%%%%%%%%%%%%%%%%%%%%%%%%%%%%%%%%%%%%%%%%%%%%%%%%%
%	Type rules
%%%%%%%%%%%%%%%%%%%%%%%%%%%%%%%%%%%%%%%%%%%%%%%%%%%%%%%%%%%
\newcounter{typerule}
\crefname{typerule}{rule}{rules}

\newcommand{\typeruleInt}[5]{%									    % #1 is the title, #2 is the hypotheses. #3 is the thesis, #4 is the label for referencing
	\def\thetyperule{#1}%
	\refstepcounter{typerule}%
	\label{tr:#4}%
  \ensuremath{\begin{array}{c}#5 \inference{#2}{#3}\end{array}} 
}
\newcommand{\typerule}[4]{%									        % #1 is the title, #2 is the hypotheses. #3 is the thesis, #4 is the label for referencing
  \typeruleInt{#1}{#2}{#3}{#4}{\textsf{\scriptsize ({#1})} \\      }
}

%%%%%%%%%%%%%%%%%%%%%%%%%%%%%%%%%%%%%%%%%%%%%%%%%%%%%%%%%%%
%	Contextual equivalence
%%%%%%%%%%%%%%%%%%%%%%%%%%%%%%%%%%%%%%%%%%%%%%%%%%%%%%%%%%%
\DeclareMathOperator\niff{\ensuremath{\nLeftrightarrow}}
\DeclareMathOperator\nsimeq{\ensuremath{\mathrel{\not\simeq}}}

\DeclareMathOperator\ceq{\ensuremath{\mathrel{\simeq_{\mi{ctx}}}}}
\DeclareMathOperator\nceq{\mathrel{\nsimeq_{\mi{ctx}}}}

\DeclareMathOperator\ceqs{\src{\ceq}}
\DeclareMathOperator\ceqt{\trgb{\ceq}}
\DeclareMathOperator\ceqo{\oth{\ceq}}

\DeclareMathOperator\nceqs{\src{\nceq}}
\DeclareMathOperator\nceqt{\trgb{\nceq}}
\DeclareMathOperator\nceqo{\oth{\nceq}}

%%%%%%%%%%%%%%%%%%%%%%%%%%%%%%%%%%%%%%%%%%%%%%%%%%%%%%%%%%%
% Missing envs
%%%%%%%%%%%%%%%%%%%%%%%%%%%%%%%%%%%%%%%%%%%%%%%%%%%%%%%%%%%
\theoremstyle{definition}
\newtheorem{assumption}{Assumption}
\newtheorem{notation}{Notation}
\newtheorem{definition}{Definition}
\newtheorem{theorem}{Theorem}
\newtheorem{lemma}{Lemma}
\newtheorem{property}{Property}
\newtheorem{example}{Example}
\newtheorem{informal}{Informal definition}
\newtheorem{corollary}{Corollary}

\Crefname{corollary}{Corollary}{Corollaries}
\Crefname{informal}{Definition}{Definition}
\Crefname{assumption}{Assumption}{Assumptions}
\crefname{assumption}{Assumption}{Assumptions}
\Crefname{property}{Property}{Properties}
\crefname{property}{Property}{Properties}
\Crefname{lstlisting}{Listing}{Listings}
\Crefname{problem}{Problem}{Problems}
\Crefname{equation}{Rule}{Rules}

%%%%%%%%%%%%%%%%%%%%%%%%%%%%%%%%%%%%%%%%%%%%%%%%%%%%%%%%%%%
% Lambda 
%%%%%%%%%%%%%%%%%%%%%%%%%%%%%%%%%%%%%%%%%%%%%%%%%%%%%%%%%%%
\DeclareMathOperator\op{\ensuremath{\oplus}}

\newcommand{\lam}[2]{\ensuremath{\lambda #1\ldotp #2}}
\newcommand{\pair}[1]{\ensuremath{\left\langle#1\right\rangle}}
\newcommand{\projone}[1]{\ensuremath{#1.1}}
\newcommand{\projtwo}[1]{\ensuremath{#1.2}}
\newcommand{\caseof}[3]{\ensuremath{{case}~#1~{of}~\inl{x_1}\mapsto #2\mid\inr{x_2}\mapsto #3}}
\newcommand{\inl}[1]{\ensuremath{{inl}~#1}}
\newcommand{\inr}[1]{\ensuremath{{inr}~#1}}
\newcommand{\fold}[1]{\ensuremath{{fold}_{#1}}}
\newcommand{\unfold}[1]{\ensuremath{{unfold}_{#1}}}
\newcommand{\Lam}[2]{\ensuremath{\Lambda #1\ldotp #2}}
\newcommand{\tapp}[2]{\ensuremath{#1 \hole{#2}}}
\newcommand{\pack}[3]{\ensuremath{{pack}~\pair{#1,#2}~{as}~#3}}
\newcommand{\unpack}[4]{\ensuremath{{unpack}~#1~{as}~\pair{#2,#3}~{in}~#4}}

\newcommand{\type}[3]{\ensuremath{ \left\{#1:#2\relmiddle|#3 \right\}}}

\newcommand{\matgen}[2]{\ensuremath{\mu #1\ldotp#2}}
\newcommand{\mat}[0]{\matgen{\alpha}{\tau}}
\newcommand{\fatgen}[2]{\ensuremath{\forall #1\ldotp#2}}
\newcommand{\fat}[0]{\fatgen{\alpha}{\tau}}
\newcommand{\eatgen}[2]{\ensuremath{\exists #1\ldotp#2}}
\newcommand{\eat}[0]{\eatgen{\alpha}{\tau}}
% \newcommand{\fatgent}[2]{\ensuremath{\trgb{\forall} #1\ldotp#2}}
% \newcommand{\fatt}[0]{\fatgent{\alpt}{\tat}}
% \newcommand{\eatgent}[2]{\ensuremath{\trgb{\exists} #1\ldotp#2}}
% \newcommand{\eatt}[0]{\eatgent{\alpt}{\tat}}

\newcommand{\fail}[0]{\mi{fail}}

\newcommand{\redgen}[1]{\ensuremath{ \hookrightarrow^{#1} }}
\newcommand{\nredgen}[1]{\ensuremath{\not\hookrightarrow^{#1}}}
\DeclareMathOperator\red{\redgen{}}

\newcommand{\nred}[0]{\nredgen{}}
\newcommand{\redstar}[0]{\redgen{*}}

\newcommand{\bigredgen}[1]{\ensuremath{ \Downarrow^{#1} }}
\newcommand{\nbigredgen}[1]{\ensuremath{\not\Downarrow^{#1}}}
\DeclareMathOperator\bigs{\bigredgen{}}

\newcommand{\credgen}[1]{\ensuremath{ \leadsto^{#1} }}
\newcommand{\ncredgen}[1]{\ensuremath{\not\leadsto^{#1}}}
\DeclareMathOperator\cred{\credgen{}}
\DeclareMathOperator\credp{\credgen{p}}

\newcommand{\subst}[2]{\ensuremath{\bl{\left[#1\relmiddle/#2\right]}}} %replace 1 in place of 2
\newcommand{\subs}[2]{\subst{\src{#1}}{\src{#2}}}
\newcommand{\subt}[2]{\subst{\trg{#1}}{\trg{#2}}}
\newcommand{\subo}[2]{\subst{\oth{#1}}{\oth{#2}}}

\usepackage{amssymb}
\usepackage{hyperref}
\usepackage{multirow}
\usepackage[most]{tcolorbox}
\tcbuselibrary{skins,breakable}

\newtcolorbox{callout}[2][]{breakable,sharp corners, skin=enhancedmiddle jigsaw,parbox=false,
boxrule=0mm,leftrule=2mm,boxsep=0mm,arc=0mm,outer arc=0mm,attach title to upper,
after title={.\ }, coltitle=purple,colback=purple!10,colframe=purple, title={#2},
fonttitle=\bfseries,#1}

\newtcolorbox{warning}[2][]{breakable,sharp corners, skin=enhancedmiddle jigsaw,parbox=false,
boxrule=0mm,leftrule=2mm,boxsep=0mm,arc=0mm,outer arc=0mm,attach title to upper,
after title={!\ }, coltitle=red,colback=red!10,colframe=red, title={#2},
fonttitle=\bfseries,#1}

\newtcolorbox{esempio}[2][]{breakable,sharp corners, skin=enhancedmiddle jigsaw,parbox=false,
boxrule=0mm,leftrule=2mm,boxsep=0mm,arc=0mm,outer arc=0mm,attach title to upper,
after title={:\ }, coltitle=blue,colback=blue!10,colframe=blue, title={#2},
fonttitle=\bfseries,#1}

\newcommand{\na}[0]{\ensuremath {\overset{N}{\rightarrow}}}
\newcommand{\rl}[3]{\inference{#1}{#2}\text{ #3}}
\newcommand{\bop}[0]{\ensuremath\oplus}
\newcommand{\appl}[2]{\ensuremath(#1)\ #2}
\newcommand{\st}[3][]{\ensuremath{\displaystyle\frac{#3\hfill}{#2\hfill} \text{#1}}}
\newcommand{\N}{\ensuremath \mathbb N}
\newcommand{\I}{\ensuremath \mathbb I}
\newcommand{\lam}[2]{\ensuremath{\lambda#1.#2}}
\newcommand{\inl}[0]{\ensuremath{\ inl\ }}
\newcommand{\inr}[0]{\ensuremath{\ inr\ }}
\newcommand{\case}[3]{\ensuremath{\text{case}#1\ \text{of}\ \left|\begin{aligned}& #2\\ & #3\end{aligned}\right.}}
\newcommand{\Da}[0]{\ensuremath{\Downarrow}}
\newcommand{\while}[2]{\ensuremath{\text{while }#1\text{ do }#2\text{ end}}}
\newcommand{\for}[3]{\ensuremath{\text{for }i=#1\text{ to }#2\text{ do }#3\text{ end}}}
\newcommand{\mE}[0]{\ensuremath{\mathbb{E}}}
\newcommand{\pair}[1]{\ensuremath{\langle#1\rangle}}
\newcommand{\V}{\ensuremath{\mathcal{V}}}
\newcommand{\cE}{\ensuremath{\mathcal{E}}}
\newcommand{\cD}{\ensuremath{\mathcal{D}}}
\newcommand{\cF}{\ensuremath{\mathcal{F}}}
\newcommand{\IF}[0]{\ensuremath {\text{ if }}}
\newcommand{\THEN}[0]{\ensuremath {\text{ then }}}
\newcommand{\ELSE}[0]{\ensuremath {\text{ else }}}
\newcommand{\AND}[0]{\ensuremath {\text{ and }}}
\newcommand{\OR}[0]{\ensuremath {\text{ or }}}
\newcommand{\unpack}[3]{\ensuremath{\text{unpack } #1 \text{ as }\langle #2 \rangle\text{ in }#3}}
\newcommand{\pack}[2]{\ensuremath{\text{pack } \pair{#1} \text{ as } #2 }}
\newcommand{\te}[1]{\text{#1}}
\newcommand{\ls}[0]{\ensuremath{\leadsto^{*}}}
\newcommand{\LET}[0]{\ensuremath{\text{ let }}}
\newcommand{\TIN}[0]{\ensuremath{\text{ in }}}
\newcommand{\NEW}[0]{\ensuremath{\text{ new }}}


\newcommand{\na}[0]{\ensuremath {\overset{N}{\rightarrow}}}
\newcommand{\rl}[3]{\inference{#1}{#2}\text{ #3}}
\newcommand{\bop}[0]{\ensuremath\oplus}
\newcommand{\appl}[2]{\ensuremath(#1)\ #2}
\newcommand{\st}[3][]{\ensuremath \displaystyle\frac{#3\hfill}{#2\hfill} \text{#1}}
\newcommand{\N}{\ensuremath \mathbb N}
\newcommand{\mE}{\ensuremath \mathbb E}
\newcommand{\V}{\ensuremath{\mathcal{V}}}
\newcommand{\cE}{\ensuremath{\mathcal{E}}}
\newcommand{\te}[1]{\text{#1}}
\usepackage{pdflscape}

\newcommand{\IF}[0]{\ensuremath {\text{ if }}}
\newcommand{\THEN}[0]{\ensuremath {\text{ then }}}
\newcommand{\ELSE}[0]{\ensuremath {\text{ else }}}
\newcommand{\AND}[0]{\ensuremath {\text{ and }}}
\newcommand{\OR}[0]{\ensuremath {\text{ or }}}
\newcommand{\LET}[0]{\ensuremath {\text{ let }}}
\newcommand{\TIN}[0]{\ensuremath {\text{ in }}}

\title{Appunti Semantics}
\author{Diego Oniarti}
\date{Anno 2024-2025}

\begin{document}

\maketitle
\tableofcontents

\section{Lambda Calculus}
Modello formale per il calcolo funzionale. \\
Il "While Language"(?) è più o meno la stessa cosa ma per la programmazione procedurale, che non faremo.

\subsection{Sintassi}
Sintassi per l'Untyped Lambda Calculus (ULC):

\begin{align*}
t :=  & n\in \mathbb{N} \\
      & | t \oplus t \\
	  & | \lambda x. t \\
	  & | x\in X \\
	  & | t\ t \\
\end{align*}

dove:
\begin{itemize}
    \item \textit{t} è una metabariabile
    \item \textit{:=} è "RNF" (?)
    \item $\oplus$ è +, -, e $\times$
    \item $\lambda$ indica una funzione, in questo caso con parametro $x$ e body $t$.
    \item Tutto è associativo a sinistra
\end{itemize}

Questo vuol dire che un termine nel nostro linguaggio è un numero naturale o una somma di termini.

\begin{callout}{nb}
    Possiamo fare delle semplificazioni come usare $n$ per rappresentare i numeri reali invece che preoccuparci della rappresentazione binaria.
\end{callout}

\begin{esempio}{example}
    $(\lambda x.x+1)\ 3$
    Questo rappresenta una funzione "successivo" e invoca la funzione sul numero $3$.
\end{esempio}

\section{SOS - Structural Operational Semantics}
\begin{align*}
t ::=& n \\
    |& t \oplus t \\
    |& \lambda x.t \\
    |& x\in X \\
    |& t\ t
\end{align*}

\begin{align*}
    \overbrace{\Omega}^\text{progrm state} :: =& t \\
                                               |&fail \\
\end{align*}

We can divide terms in \textbf{redexes} and \textbf{values}.
\paragraph{Redexes}
\begin{itemize}
    \item $n\oplus n$
    \item $(\lambda x.t)\ v$
\end{itemize}

\paragraph{Values}
\begin{align*}
    v ::&= n \\
        &| \lambda x.t
\end{align*}

Redexes change the state of the program according to some rules:
\paragraph{rules}
\begin{align*} 
    &\frac{[n\oplus n'] = n''}{n\oplus n'\to n''} &&\text{sos-bop} \\
    &\frac{}{(\lambda x.t) v\to t[\frac v x]}\ &&\text{sos-beta} \\
    &\frac{t\to t''}{t\oplus t' \to t''\oplus t'} &&\text{sos-bop-1} \\
    &\frac{t\to t'}{n\oplus t \to n\oplus t'} &&\text{sos-bop-2} \\
    &\frac{t\to t''}{t\ t' \to t''\ t'}\ &&\text{sos-app-1} \\
    &\frac{t'\to t''}{(\lambda x.t) t'\to (\lambda x.t)\ t''}\ &&\text{sos-app-2} \\
\end{align*}

\paragraph{substitution}
\begin{align*}
    n[v/x] &= n \\
    x[v/x] &= v \\
    y[v/x] &= y \\
    \\
    (t\oplus t')[v/x] &= t[v/x]\oplus t'[v/x] \\
    (t\ t')[v/x]  &= t[v/x]\ t'[v/x] \\
    (\lambda y.t)[v/x] &= \lambda y.t[v/x] \\
\end{align*}

Ogni regola modifica lo stato del programma, quindi possiamo dire abbiano la forma $\Omega \to \Omega$. 
Un programma corretto risolve a un \textit{valore} dopo una serie di "\textit{steps}".

\paragraph{Errori}
Programmi come "$5\ 4$" o "$0 + (\lambda x.x)$" sono ben formati dal punto di vista della grammatica indicata. Portano però a delle redex a cui non di può applicare alcuna regola. \\
Aggiungiamo quindi uno stato "\textit{fail}" a $\Omega$ e delle regole per propagare questo fail.

\paragraph{Fails}
\begin{align*}
    \frac{}{(\lambda x.t)\oplus t\to fail}\ &\text{sos-f-L} \\
    \frac{}{n\ t\to fail}\ &\text{sos-f-n} \\
    \frac{}{n\ \oplus \lambda x.t\to fail}\ &\text{sos-f-L2} \\
    \\
    \frac{t\to t'' \  t''\to fail}{t\oplus t' \to fail}\ &\text{sos-bop-f1} \\
    \frac{t\to t'\ t'\to fail}{n\oplus t \to fail}\ &\text{sos-bop-f2} \\
    \frac{t\to t'' \ t''\to fail}{t\ t' \to fail}\ &\text{sos-app-f1} \\
    \frac{t'\to t''\ t''\to fail}{(\lambda x.t)\ t'\to fail}\ &\text{sos-app-f2} \\
\end{align*}

\section{SOS - Call By Name}
We don't apply a function to values but to symbols. The symbols are then lazily evaluated when they're used.
\[
    \Omega\na\Omega
\]
Let's see which rules change under these new assumption:
$$\begin{array}{c l}
    \frac{}{n\oplus n' \overset{N}{\rightarrow} n"}\ &\text{sos-bop N} \\
    \frac{}{(\lambda x.t)\ t'\na t[\frac {t'} x]}\   &\text{sos-beta N} \\
    \text{untouched}                                 &\text{sos-app1N } \\
    \text{untouched}                                 &\text{sos-bop1N } \\
    \text{untouched}                                 &\text{sos-bop2N } \\
\end{array}$$

\section{Big Step}
Una semantica \textit{big step} ha un judgement del tipo:
\[
    t\Downarrow v
\]
Questo vuol dire che le inverence rules non fatto più pattern matching su $\Omega\to\Omega$ ma su $t\Downarrow v$ (il termine $t$ riduce a un valore $v$). \\
rules:
\begin{align*}
    \frac{}{v\Downarrow v}\ &\text{ val} \\
    \frac{t\Downarrow n\ t'\Downarrow n'\ n\oplus n'=n"}{t\oplus t' \Downarrow n"}\ &\text{bs-bop} \\
    \frac{t\Downarrow \lambda x.t"\ t'\Downarrow v\ t"[v/x]\Downarrow v'}{t\ t'\Downarrow v'}\ &\text{bs-app} \\
\end{align*}

\subsection{Equivalenza con SS}
Big Step e Small Step sono equivalenti. Questo vuol dire che ogni termine che riduce a un valore in big step, converge allo stesso valore in small step. 
Questo è utile per alcune dimostrazioni, in quanto possiamo usare la struttura ad albero di BS nelle dimostrazioni per SS.

\section{Contextual Operation Semantics}
\subsection{COS, SS, CBV}
Chiamiamo $E$ l'\textit{evaluation context}, così definito.
\begin{align*}
E ::= &[] \\
|& E\ t \\
|& (\lambda x.t)E \\
|& E \oplus t \\
|& n \bop E 
\end{align*}

Abbiamo poi 2 judgements
\begin{align*}
    \Omega &\cred \Omega & \text{main reduction} \\
    \Omega &\credp \Omega & \text{primitive reduction}
\end{align*}

\begin{gather*}
    \rl{t\credp t'}{E[t]\cred E[t']}{ctx} \\
    \rl{}{n\bop n' \credp n"}{c-bop} \\
    \rl{}{(\lam{x}{t})v \credp t[v/x]}{c-beta}
\end{gather*}

\begin{callout}{esercizio}
(((\lam{x}{\lam{y}{\lam{z}{z\ x-y\ x}}})5)(\lam v v))(\lam w {2*w})
\end{callout}

\begin{callout}{wow}
SOS e COS risolvono un'espressione con lo stesso numero di passaggi
\end{callout}

\section{Teorema di equivalenza SOS e COS}
$\forall t,t'.t\to t' \iff t\cred t'$ \\
Per ogni coppia di termini $t$ e $t'$, $t$ fa uno step SOS a $t'$ se e solo se $t$ fa anche uno step COS a $t'$.
Per dimostrare l'$iff$ dimostriamo prima il $\implies$ e poi l'$\impliedby$.

\paragraph{lem.1} $\forall t,t'.t\to t' \implies t\cred t'$
\paragraph{lem.2} $\forall t,t'.t\to t' \impliedby t\cred t'$

\subsection{Prova per induzione del lemma 1}
Usiamo i termini come struttura induttiva. Se vediamo i termini come il loro Abstract Syntax Tree, possiamo partire da termini la cui altezza è zero e costruirne altri più complessi per induzione. \\
L'altra struttura induttiva che possiamo usare è la derivazione SOS. Anche essa è un albero, quindi lo stresso ragionamento vale.

Iniziamo quindi con i casi base. In questo caso abbiamo solo \textit{bop} e \textit{beta}.
\begin{itemize}
    \item BOP 
        \begin{align*}
            \quad & t = n\bop n'\quad t'=n" \\
            \text{TS:}  \quad & n\bop n' \cred n" \\
            \quad & \text{by ctx with } E = [] \\
            \text {TS:} \quad & n\bop n'\credp n" \\
            \quad & \text{by c-bop}
        \end{align*}
    \item BETA
        \begin{align*}
            \quad & t = (\lam{x}{t"})v\quad t'=t"[v/x] \\
            \text{TS:}  \quad & (\lam{x}{t"})v \cred t"[v/x] \\
            \quad & \text{by ctx with } E = [] \\
            \text {TS:} \quad & (\lam{x}{t"})v \credp t"[v/x] \\
            \quad & \text{by c-beta}
        \end{align*}
\end{itemize}
Dimostriamo ora il passo induttivo per la prova del della 1: \\
In questo caso avremmo 4 casi induttivi da dimostrare (bop1, bop2, app1, app2) ma ne facciamo uno (app1) solo per brevità.
\[
    \text{TH:} \quad \forall t_h,t_h'\ if\ t_h\to t_h'\ then\ t_h\cred t_h' \\
\]
\begin{itemize}
    \item app1: $t=t_1\ t_2 \quad t'=t_1'\ t_2$
        \begin{align*}
            \text{TH:}\quad  & t_1\ t_2 \cred t_1'\ t_2 \\
            \text{HP1:}\quad & t_1\ t_2 \to t_1'\ t_2 \\
            \text{HP2:}\quad & t_1 \to t_1' \\
                             & \text{by IH with HP2 wh } t_1\cred t_1' \quad \text{HT1} \\
                             & \text{by inversion on HT1 wh }
                             \begin{cases} 
                                 t_1 \equiv E[t_0]   &  \text{HE1} \\
                                 t_1'\equiv E[t_0']  & \text{HE1'} \\
                                 t_0 \credp t_0'     & \text{HPR} \\
                             \end{cases} \\
                             & \text{by HE1, HE1' TS } E[t_0]\ t_2\cred E[t_0']\ t_2 \quad^{(*)}\\
                             & \text{by ctx} \\
                             & \text{with } E'=E\ t_2\text{ and HPR} \\
                             & E[t_0]\ t_2 \equiv E'[t_0] \cred E'[t_0'] \quad^{(*)}
        \end{align*}
\end{itemize}

\subsection{Prova per definizione del lemma 2}
\[
    \forall t,t'.\ t\cred t' \implies t\to t'
\] 
\subparagraph{lemma a} $\forall t,t'.\ t\to t' \implies E[t]\to E[t']$

\subparagraph{lemma b} $\forall t,t'.\ t\credp t' \implies t\to t'$

\begin{align*}
    \text {by inversion on HP} \quad& t\equiv E[t_0]  & HE0 \\
                         \quad& t'\equiv E[t_0']  & HE0' \\
                         \quad& t_0\credp t_0'  & HPR \\
    \text{by LB with HPR w.h.}\quad& t_0\to t_0' & HR \\
    \text{by HE0,HE0' T.S.} \quad& E[t_0] \to E[t_0']  \\
    \text{by LA with HR} \quad& \text{the thesis holds}
\end{align*}

\subparagraph{Proof Lemma B} Proof by case study on $\credp$
\subparagraph{Proof Lemma A} Proof by induction on $E$
\begin{itemize}
    \item Base
        \begin{align*}
            E = [] \\
            TS t\to t' & \text{by HP} 
        \end{align*}
    \item Induzione. 
        \begin{itemize}
            \item IH: $t\to t' \implies E'[t]\to E'[t']$
            \item $E = E'[t"]$
            \item by IH with HP.$E'$ w.h. $E'[t]\to E'[t']$
            \item TS $(E'\ t")[t] \to ()$
        \end{itemize}
\end{itemize}

\section{Simply Typed Lambda Calculus}
I programmi descritti dal STLC sono un subset di tutti i programmi descritti dal ULC. \\
STLC non descrive però l'insieme di \textbf{tutti} i programmi che non falliscono. I \textit{type system} fanno una over-approssimazione, rifiutando alcuni programmi che potrebbero ridurre a un valore. \\
In fine, un programma STLC può ancora divergere (finire in un loop infinito).
\begin{esempio}{Progranna ULC non STLC che non fallisce}
    $$(\lambda x.0)(\lambda y.3+\lambda z.z)$$
    Il programma, assumendo call by name, riduce correttamente a $0$. Questo è un comportamento che si può apprezzare a run time, ma non a compile time (dove vive il \textit{type system}).
\end{esempio}

\paragraph{Tipi}
\begin{align*}
    \tau :=&N \\
           &\tau\to\tau
\end{align*}

\paragraph{Judgment}
\begin{align*}
    vedi\ foto
\end{align*}

\paragraph{recap}
\subparagraph{temini}
\begin{align*}
    t := &n \\
         &t\bop t \\
         &\lam{x:\tau}{t} \\
         &x \\
         &t\ t \\
\end{align*}

\subparagraph{v}
\begin{align*}
    v := &n \\
         &\lam{x:\tau}{t} \\
\end{align*}

\subparagraph{tipi}
\begin{align*}
    \tau :=&N \\
           &\tau\to\tau
\end{align*}

\subparagraph{typing environment}
\begin{align*}
    \Gamma := &\emptyset \\
              &\Gamma,x:\tau
\end{align*}

\section{Expanding The STLC}
\subsection{Aggiungere tuple}
\begin{align*}
    t := &\dots \\
    |&<t,t> \\
    |&t.1 \\
    |&t.2 \\
    \\
    \tau:= &\dots \\
    |&\tau\times\tau \\
    \\
    v:=&\dots \\
    |&<v,v> \\
    \\
    E := &\dots \\
    |&<E,t> \\
    |&<v,E> \\
    |&E.1 \\
    |&E.2 \\
    \\
    \frac{}{<v_1,v_2>.1\credp v_1}p1 & \frac{}{<v_1,v_2>.2\credp v_2}p2
\end{align*}

\subsection{Aggiungere inums}
\begin{align*}
    t := &\dots \\
    |&inl\ t \\
    |&inr\ t \\
    |&case\ t\ of\ inl\ x\mapsto t | inr\ x\mapsto t \\
    \\
    \tau:= &\dots \\
    |&\tau_1 \cup+ \tau_2 \\
    \\
    v:=&\dots \\
    |&inl\ v \\
    |&inr\ v \\
\end{align*}
\begin{align*}
    E := &\dots \\
    |&inl\ E \\
    |&inr\ E \\
    |&case\ t\ of\ inl\ x\mapsto t| inr\ x\mapsto t \\
    \\
     & \frac{}{case\ inl\ v\ of\ inl\ x_1\mapsto t_1| inr\ x_2\mapsto t_2\credp t_1[v/x_1]}inL \\
     & \frac{}{case\ inr\ v\ of\ inl\ x_1\mapsto t_1| inr\ x_2\mapsto t_2\credp t_2[v/x_2]}inR
\end{align*}

\subsection{Booleani}
Ci sono due modi in cui potremmo aggiungere booleani nel linguaggio.
\begin{itemize}
    \item true: $\lambda x.\lambda y. x$
    \item false: $\lambda x.\lambda y. y$
    \item $if\ t\ then\ t_1\ else\ t_2$ $t\ t_1\ t_2$
\end{itemize}
Questo fa evaluation sia di $t_1$ che $t_2$. Possiamo risolvere così:
\begin{itemize}
    \item true: $\lambda x.\lambda y. x\ 0$
    \item false: $\lambda x.\lambda y. y\ 0$
    \item $if\ t\ then\ t_1\ else\ t_2$ $t\ (\lambda \_.t_1)\ (\lambda\_.t_2)$
\end{itemize}
Oppure così:
\begin{itemize}
    \item true: $\lambda x.\lambda y. x$
    \item false: $\lambda x.\lambda y. y$
    \item $if\ t\ then\ t_1\ else\ t_2$ $(t\ (\lambda \_.t_1)\ (\lambda\_.t_2))0$
\end{itemize}

\begin{landscape}
    \begin{align*}
        \st[lam]{
            \emptyset: \lam{x:\N\to \N\to \N}{\lam{y:\N\to \N}{\lam{a:\N}{\lam{b:\N}{x\ (y\ a)\ (y\ b)}}}}:(\N\to\N\to\N)\to((\N\to\N)\to(\N\to(\N\to\N)))
        }{
            \st[lam]{
                x:\N\to\N\to\N\vdash \lam{y:\N\to \N}{\lam{a:\N}{\lam{b:\N}{x\ (y\ a)\ (y\ b)}}} : (\N\to\N)\to(\N\to(\N\to\N))
            }{
                \st[lam]{
                    \substack{x:\N\to\N\to\N,\hfill\\y:\N\to\N\hfill}\vdash \lam{a:\N}{\lam{b:\N}{x\ (y\ a)\ (y\ b)}}:\N\to(\N\to\N)
                }{
                    \st[lam] {
                        \substack{x:\N\to\N\to\N,\hfill\\y:\N\to\N,\hfill\\a:\N\hfill}\vdash \lam{b:\N}{x\ (y\ a)\ (y\ b)}:\N\to\N
                    }{
                        \st[app] {
                            \Gamma \left\{\substack{x:\N\to\N\to\N,\hfill\\y:\N\to\N,\hfill\\a:\N,\hfill\\b:\N\hfill}\right.\vdash x\ (y\ a)\ (y\ b):\N
                        }{
                            \st[app]{
                                \Gamma\vdash x\ (y\ a): \N\to\N
                            }{
                                \st[val]{
                                    \Gamma\vdash x:\N\to\N\to\N
                                }{
                                    \Gamma(x)=\N\to\N\to\N
                                }
                                \quad
                                \st[app]{
                                    \Gamma\vdash y\ a:\N
                                }{
                                    \st[var]{
                                        \Gamma\vdash y:\N\to\N
                                    }{
                                        \Gamma(y)=\N\to\N
                                    }
                                    \quad
                                    \st[var]{
                                        \Gamma\vdash a:\N
                                    }{
                                        \Gamma(a)=\N
                                    }
                                }
                            }
                            \quad
                            \st[app]{
                                \Gamma\vdash y\ b:\N
                            }{
                                \st[var]{
                                    \Gamma\vdash y:\N\to\N
                                }{
                                    \Gamma(y)=\N\to\N
                                }
                                \quad
                                \st[var]{
                                    \Gamma\vdash b:\N
                                }{
                                    \Gamma(b)=\N
                                }
                            }
                        }
                    }
                }
            }
        }
    \end{align*}
    \begin{align*}
        \st[app]{
            \emptyset\vdash(\lam{x:\N}{2*x})\ 5:\N
        }{
            \st[lam]{
                \emptyset\vdash \lam{x:\N}{2*x}:\N\to\N
            }{
                \st[num]{
                    x:\N\vdash 2*x:\N
                }{}
            }
            \quad
            \st[num]{
                \emptyset\vdash 5: \N
            }{}
        }
    \end{align*}
\end{landscape}

\section{If Then Else}
Assumiamo questo encoding per \textit{true} e \textit{false}:
\begin{align*}
    True &= inl0 & Bool = \N\uplus\N \\
    False &= inr 1 \\
\end{align*}
\begin{align*}
    if\ t\ then\ t' = 
\end{align*}

\section{Properties of STLC}
\subsection{Type soundness}
\begin{align*}
   & if\ \emptyset\vdash t:\tau\ and\ t\cred^* t'\ then\ either \\
   & \vdash t.VAL \\
   & or \\
   & \exists t".t'\cred t"
\end{align*}
Se abbiamo un termine \textit{well typed}, prima o poi riduce a un valore o a un termine che può ancora ridurre.

\begin{callout}{star-step}
    \begin{align*}
        \st{t\cred^* t}{}\quad\st{t\cred^* t'}{t\cred t"\quad t"\cred^* t'}
    \end{align*}
\end{callout}

\subsubsection{Progress}
\begin{align*}
    & if\ \emptyset\vdash t.\tau\ then\ either \\
    & \vdash t.VAL\ or \\
    & \exists t'.t\cred t'
\end{align*}

\subsubsection{Preservation}
\begin{align*}
    if\ \emptyset\vdash t.\tau\ and\ t\cred t'\ then\ \emptyset\vdash t'.\tau
\end{align*}

\subparagraph{Lem: Canonicity}
\begin{align*}
    & if\ \Gamma\vdash v.N & then\quad & v=n &\\
    & if\ \Gamma\vdash v.\tau\to\tau' & then\quad & v=\lam{x:\tau}{t'} &\\
    & if\ \Gamma\vdash v.\tau\times\tau' & then\quad & v=<v_1,v_2> &\\
    & if\ \Gamma\vdash v.\tau\uplus\tau' & then\quad & v=...
\end{align*}

\subsection{Normalization}
\begin{align*}
    if\ \emptyset\vdash t.\tau\ then \ \exists v. t\cred^* v
\end{align*}

\subsection{proofs}
\subsubsection{Proof of Progress}
\begin{align*}
    & if\ \emptyset\vdash t.\tau\ then\ either \\
    & \vdash t.VAL\ or \\
    & \exists t'.t\cred t'
\end{align*}
Proof by induction on the typing derivation.
\subparagraph{Base} 
\begin{itemize}
    \item t.VAR
        \begin{align*}
            \st[contradiziona]{\emptyset\vdash x.\tau}{\emptyset(x)=\tau}
        \end{align*}
    \item t.NAT
        \begin{align*}
            \st{\emptyset\vdash n.\N}{}
        \end{align*}
        TS either $\vdash n.VAL$ or $\exists \tau'. n\cred t'$
\end{itemize}

\subparagraph{Induction}
\begin{itemize}
    \item T-lam
        \begin{align*}
            \st{\emptyset\vdash \lam{x:\tau}{t'}:\tau\to\tau'}{}
        \end{align*}
        TS either $\underline{\vdash \lam{x:\tau}{t}.VAL}$ or $\exists ...$
    \item T-app
        \begin{align*}
            \st{\emptyset\vdash t'\ t":\tau}{\emptyset\vdash t':\tau'\to\tau \quad \emptyset\vdash t":\tau'}
        \end{align*}
\end{itemize}

\subsubsection{Proof of Preservation}
Assumendo $t\equiv E[t_0]$, abbiamo il judgment $\vdash E:\tau\to\tau$
\begin{align*}
    & \st[et-hole]{\vdash[\cdot]:\tau\to\tau}{} \\
    & \st[et-app]{\vdash E\ t:\tau\to\tau'}{\vdash E:\tau\to(\tau"\to\tau')\quad \emptyset\vdash t:\tau"} \\
    & \st[et-lam]{\vdash (\lam{x:\tau}{t}) E:\tau"\to\tau'}{\emptyset\vdash (\lam{x:\tau}{t}):\tau\to\tau'\quad\vdash E:\tau"\to\tau} \\
    & \st[et-bopp]{\vdash E\bop t:\tau\to\N}{\vdash E:\tau\to\N\quad\emptyset\vdash t:\N} \\
    & \st[et-bopp]{\vdash n\bop E:\tau\to\N}{\emptyset\vdash n:\N\quad\vdash E:\tau\to\N} \\
\end{align*}

\paragraph{Primitive Preservation} $if\ \emptyset\vdash t:\tau\ and\ t\credp t'\ then\ \emptyset\vdash t'.\tau$
\subparagraph{proof} Casa analisys on $\credp$

\paragraph{Decomposition} $if\ \emptyset\vdash E[t]:\tau\ then\ \exists\tau'.\vdash E:\tau'\to\tau\ and\ \emptyset\vdash t:\tau'$
\subparagraph{Proof} induction on $E$

\paragraph{Composition} $if\ \vdash E: \tau\to\tau'\ and\ \emptyset\vdash t:\tau\ then\ \emptyset\vdash E[t]:\tau'$
\subparagraph{Proof} by induction on $\vdash E:\tau\to\tau'$

\begin{align*}
    \text{by inversion on HP} & t\equiv E[t_0] & HT0 \\
                              & t'\equiv E[t_0'] & HT1 \\
                              & t_0\credp t_0' & HTP \\
    \text{by HT0 to HP1 with }& \emptyset\vdash E[t_0]:\tau & HP1N \\
    \text{by HT1 to TH. TS}   & \emptyset\vdash E[t_0']:\tau \\
    \text{by decomposition with HP1N w.h.} & \vdash E:\tau'\to\tau & HE \\
                                           & \emptyset\vdash t_0:\tau'&HTT0 \\
    \text{by prim. pres with HTT0 and HTP w.h} & \emptyset\vdash t_0':\tau' & HTT1 \\
    \text{by compos with HE and HTT1 W.h.} & \emptyset\vdash E[t_0']:\tau & HF \\
    \text{by HF the thesis holds}
\end{align*}

\subsubsection{Proof of Normalization}
\begin{align*}
    if \emptyset\vdash t:\tau\ then\ \exists v.t\cred^* v
\end{align*}
\paragraph{Proof} by induction on T.D of $t$ \\
\begin{itemize}
    \item base 
    \item induction \begin{itemize}
            \item $t=t_1\ t_2\quad \st{\emptyset\vdash t_1\ t_2:\tau}{\emptyset\vdash t_1:\tau'\to\tau\quad\emptyset\vdash t_2:\tau'}$
        \end{itemize}
\end{itemize}
Questo non possiamo provarlo con gli strumenti che abbiamo fin ora. Serve quindi introdurre le relazioni logiche.

\section{Logical Relationships (and semantic typing)}
\begin{align*}
    \V\left[\tau\right] & \text{Quali valori costituiscono un tipo} \\
    E\left[\tau\right] & \text{Quali termini costituiscono un tipo}\\
    G\left[\Gamma\right] & \text{Sostituzione} \\
    \gamma ::=&\emptyset \\
    |&\gamma[v/x] 
\end{align*}

\paragraph{Def SemTy (semantic typing)}: \begin{align*}
    \Gamma \vDash t:\tau \hat{=} \forall \gamma\in G[\tau].t\gamma\in \cE[\tau] 
\end{align*}

\paragraph{Semantic soundness}
\begin{align*}
    if\ \Gamma\vdash t:\tau\ then\ \Gamma \vDash t:\tau
\end{align*}
Se un programma è well typed in syntactic typing, lo è anche in semantic typing.

\begin{align*}
    \mathcal{V}[\N] &= \{n\} \OR \V[\N] = \{v|v\equiv n\} \\
    \V[\tau\to\tau'] &= \{v|v\equiv \lam{x:\tau}{t} \AND \forall v' \IF v'\in \V[\tau] \THEN t[v'/x] \in \cE[\tau']\} \\
    \V[\tau\times\tau'] &= \{v|v\equiv \pair{v_1,v_2} \AND t \in \V[\tau] \AND t'\in \V[\tau']\} \\
    \V[\tau\uplus\tau'] &= \{v|v\equiv \inl v_1 \AND v_1\in \V[\tau]\} \cup \{v|v\equiv \inr v_1 \AND v_1\in \V[\tau']\} \\
    \cE[\tau]&=\{t|\exists v.t\cred^* v \AND v\in \V[\tau] \} \\
    G[\emptyset] &= \emptyset \\
    G[\Gamma, x:\tau] &= \{\gamma[v/x] | \gamma\in G[\Gamma] \AND v\in\V[\tau] \}
\end{align*}

\section{Proof of Normalization}
\begin{align*}
    proof\ by\ SS\ w.h\ & \emptyset \vDash t.\tau \\
    ...
\end{align*}


first projection
$t=t_1$
\begin{align*}
    \Gamma\vDash \tau\times\tau'\ and\ 
\end{align*}

\section{lemma: vals in terms}
\begin{align*}
    \forall t\ if\ t\in V[\tau]\ then\ t\in E[\tau] 
\end{align*}

\section{Compatibility lemmas}
\subsection{Application}
\begin{align*}
    if \Gamma\vDash t_1:\tau\to\tau'\ and\ \Gamma\vDash t_2:\tau\ then\ \Gamma\vDash t_1\ t_2:\tau'
\end{align*}
\paragraph{proof}
\begin{align*}
    & by\ def\ s.t\ take\ \gamma\in G[\Gamma]\ t.s\ (t_1\ t_2)\gamma\in E[\tau'] \\
    & by\ def\ s.t\ with\ HP1\ wh\ t_1\gamma\in E[\tau\to\tau'] \\
    & by\ def\ E\ \exists v_1.(t_1\gamma)\cred^* v_1\ and\ v_1\in V[\tau\to\tau'] \\
    & \dots by\ def\ V\ v_1\equiv \lam{x:\tau}{t_1'}\ and\ \forall v_1'\IF v_1'\in V[\tau]\THEN t_1'[v_1'/x]\in E[\tau'] \\
    & by\ def\ s.t\ with\ HP2\ wh\ t_2\gamma\in E[\tau]\ by\ def\ E\ \exists v_2. (t_2\gamma)\cred^*v_2 \AND v_2\in V[\tau] \\
    & (t_1\ t_2)\gamma = (t_1\gamma)(t_2\gamma) \\
    % & by\ HR1,HR2\ wh\ (t_1\ t_2)\gamma\cred^* v_1(t_2\gamma)\cred^*v_1\ v_2\equiv (\lam{x:\tau}{t\ t_1)v_2\cred t_1[v_2/x] 
\end{align*}

\section{Introduction and Destruction}
Le regole del linguaggio semantico possono essere divise in \textit{introduzioni} e \textit{eliminazioni}

\begin{align*}
    & \st[introduzione]{\Gamma\vDash \tau{x:\tau}{t}:\underline{\tau\to\tau'}}{\Gamma,x:\tau\vDash t:\tau'} \\
    \\
    & \st[distruzione]{\Gamma\vDash t_1\ t_2:\tau_1}{\Gamma\vDash t_1:\underline{\tau\to\tau_1}\quad \Gamma\vDash t_2:\tau}
\end{align*}

\subsection{logica}
\begin{align*}
    \st[$\implies$E]{B}{A\quad A\implies B} \\
    \st[$\implies$I]{A\implies B}{\displaystyle\overset{\displaystyle\overset{[A]}{\vdots}}{B}} \\
    \st[$\wedge$ I]{A\wedge B}{A\quad B} \\
    \st[AE1]{A}{A\wedge B} \\
    \st[AE2]{B}{A\wedge B} \\
\end{align*}

\section{System F}
\begin{align*}
    t :=& ... \\
    |& \Lambda\alpha.t \\
    |& t[\tau] \\
    \\
    \tau :=& ... \\
    |& \forall \alpha.\tau \\
    |& \alpha \\
    \\
    v :=& ... \\
    |& \Lambda\alpha.t \\
    \\
    E :=& ... \\
    |& E[\tau] \\
    \\
    \Delta :=& \emptyset \\
    |& \Delta,\alpha \\
    \\
    \Gamma :=& \emptyset \\
    |& \Gamma,x:\tau
\end{align*}

\begin{align*}
    \st[$big\beta$]{(\Lambda\alpha t)[\tau] \credp t[\tau/\alpha]}{}
\end{align*}

Nuovo typing judgment:
\begin{align*}
    \Delta,\Gamma\vdash t:\tau
\end{align*}

Syntactic type checking:
\begin{align*}
    \st{\Delta,\Gamma\vdash \Delta\alpha t:\forall\alpha.\tau}{\Delta}
\end{align*}

\begin{gather*}
    \st{\Delta\vdash\N}{} \\
    \st{\Delta\vdash\tau\to\tau'}{\Delta\vdash\tau\quad \Delta\vdash\tau'} \\
    ...
\end{gather*}

% \paragraph{esempio}
% \begin{multiline*}
%     \Lambda\alpha\Lambda\beta\Lambda\gamma:\lambda f:\alpha\to\beta\to\gamma.\lambda \\x:\beta.\lambda :\alpha. fyx: \forall\alpha\forall\beta\forall\gamma.(\alpha\to\beta\to\gamma)\to\beta\to\gamma\to\gamma
% \end{multiline*}
\subsection{Existential Types}

Un record con almeno due label is\_on e is\_off. 
Definire il tipo Switch e un termine di questo tipo

\section{free theorem}
\begin{align*}
    \IF 
\end{align*}

\paragraph{bool}
\begin{align*}
    \forall\alpha.\alpha\to\alpha\to\alpha \\
    T: \Lambda\alpha.\lam{t:\alpha}\lam{f:\alpha}{t} \\
    F: \Lambda\alpha.\lam{t:\alpha}\lam{f:\alpha}{f} \\
    \IF v \THEN\ v_t \ELSE v_f \equiv v[\tau]\ v_t\ v_f
\end{align*}

\section{altro system F}
\begin{align*}
    pack\ \left\langle\N,\left.\begin{cases}val=0\\ison=\lam{x:\N}{x==0}\\toggle=\lam{x:N}{\IF x==0\THEN 1 \ELSE 0}\end{cases}\right\}\right\rangle
\end{align*}

\section{STLC-$\mu$}
STLC-$\mu$ aggiunge i tipi ricorsivi.

\begin{align*}
    \tau ::= \cdots | \mu\alpha.\tau
\end{align*}

\begin{align*}
    list[nat] \overset{\Delta}{=}\mu\alpha.\underbrace{B}_\text{empty}\uplus(\N\times\alpha)
\end{align*}
This unfolds to:
\begin{gather*}
    B\uplus(\N\times \mu\alpha. B\uplus(\N\times\alpha))
\end{gather*}
And we could keep unfolding the $\alpha$ over and over.

There are two schools of thought over this topic: isorecursive and equirecursive

\subsection{isorecursive}
We assume the folded and unfolded type are isomorphic. This isomorphism is seen at the term level.
\begin{align*}
    t ::=\cdots &|fold_{\mu\alpha.\tau}t \\
                &|unfold_{\mu\alpha.\tau} t \\
    v ::= \cdots &|fold_{\mu\alpha.\tau} t
\end{align*}
Questo metodo rende il type-checking deterministico, ma aggiunge uno step di riduzione

\subsection{equirecursive}
L'equirocorsione rende il typing non deterministico ma non aggiunge step di riduzione. 

È possibile dimostrare che i due metodo sono tecnicamente equivalenti.

\subsection{Typing rule ISO}
\begin{gather*}
    \st[t-fold]{
        \Gamma\vdash fold_{\mu\alpha.\tau} t : \mu\alpha.\tau
    }{
        \Gamma\vdash t:\tau[\mu\alpha.\tau/\alpha]
    } \\
    \st[t-unfold]{
        \Gamma\vdash unfold_{\mu\alpha.\tau}t:\tau[\mu\alpha.\tau/\alpha]
    }{
        \Gamma\vdash t:\mu\alpha.\tau
    } \\
\end{gather*}

\subsection{Modelliamo una lista in ISO}
\begin{align*}
    nil &\overset\Delta=fold_{list[nat]}\inl false \\
    cons &\overset\Delta= \lam{x:\N}\lam{l:list[nat]}{fold_{list[nat]}\inr\pair{x, l}}
\end{align*}

\begin{callout}{typing derivation}
    \begin{align*}
        \emptyset\vdash\lam{x:\N}\lam{l:list[nat]}{fold_{ln}}
    \end{align*}
    Couldn't be arsed. Look at the lecture.
\end{callout}

List of generic $\alpha$:
\begin{align*}
    \forall\alpha.\mu\beta.B\uplus(\alpha\times\beta)
\end{align*}

%List of mixed generic types
%\begin{align*}
%    \mu\beta.\forall\alpha.B\uplus(\alpha\time\beta)
%\end{align*}

\begin{align*}
    cons \overset\Delta= \Lambda\beta.\lam{x:\beta}\lam{l:list[\alpha]}{fold_{list[\alpha]}\inr\pair{x, l}}
\end{align*}

\subsection{fold unfold cancellation}
\begin{gather*}
\end{gather*}

\subsection{Diverging computation}
\begin{gather*}
    K \overset\Delta= \mu\alpha.\alpha\to\alpha \\
    (\lam{x}{x\ x})(\lam{x}{x\ x}) \\
    (\lam{x:\mu\alpha.\alpha\to\alpha}{(unfold\ x)\ x})\ fold(\lam{x:\mu\alpha.\alpha\to\alpha}{(unfold\ x)\ x})
\end{gather*}

\subsection{recap isorecursion}
\begin{gather*}
    \tau ::= \cdots | \mu\alpha.\tau \\
    t ::= fold_{\mu\alpha.\tau}t | unfold_{\mu\alpha.\tau}t \\
    unfold_{\mu\alpha.\tau}fold_{\mu\alpha.\tau}v\leadsto^p v \\
    \st[t-fold]{
        \Delta;\Gamma\vdash fold_{\mu\alpha.\tau} t : \mu\alpha.\tau
    }{
        \Delta;\Gamma\vdash t:\tau[\mu\alpha.\tau/\alpha]
    } \\
    \st[t-unfold]{
        \Delta;\Gamma\vdash unfold_{\mu\alpha.\tau}t:\tau[\mu\alpha.\tau/\alpha]
    }{
        \Delta;\Gamma\vdash t:\mu\alpha.\tau
    } \\
\end{gather*}

\subsection{Equirecursion}
\begin{gather*}
    \st[t-eqi]{
        \Delta;\Gamma\vdash t:\tau
    }{
        \Delta,\Gamma \vdash t:\sigma
        \quad
        \Delta \vdash \sigma \overset o= \tau
    }
\end{gather*}
Questa regola può potenzialmente essere applicata in ogni passaggio del type-checking. Questo rende il processo non deterministico.

$\Delta\vdash \sigma\overset o= \tau$
\begin{gather*}
    \st[t-sym]{\Delta\vdash\sigma\circeq\tau}{\Delta\vdash\tau\circeq\sigma} \\
    \st[t-trans]{\Delta\vdash\sigma\circeq\tau}{\Delta\vdash\sigma\circeq\gamma\quad \Delta\vdash\gamma\circeq\tau} \\
    \st[t-base]{\Delta\vdash\tau\circeq\tau}{\tau\in\{\N,Bool,Unit\}} \\
    \st[t-bin]{\Delta\vdash\tau_1\star\tau_2\circeq\sigma_1\star\sigma_2}{\Delta\vdash\tau_1\circeq\sigma_1\quad\Delta\vdash\tau_2\circeq\sigma_2} \\
    \st[t-$\mu$]{\Delta\vdash\mu\alpha.\tau \circeq \mu\alpha.\sigma}{\Delta,\alpha\vdash \tau\circeq\sigma} \\
    \st[t-unfold]{\Delta\vdash\mu\alpha.\tau\circeq\sigma}{\Delta\vdash\tau[\mu\alpha.t/\alpha]\circeq\sigma} \\
    \st[t-var]{\Delta\vdash\alpha\circeq\alpha}{\alpha\in\Delta} \\
\end{gather*}

\subsection{Logical relations}
La vecchia term relation
\begin{align*}
    \cE[\tau]=\{t|\underbrace{\exists v.t\leadsto^* v}_\text{safety}\AND v\in\V[\tau]\}
\end{align*}
aveva un certo concetto di "safety" che non accetta l'esecuzione divergente. Quindi dobbiamo modificarlo.

Introduciamo questo judgment $t\searrow_n t'$ chiamato \textit{numbered stepping}. $t$ steppa a $t'$ in esattamente $n$ step e non può più steppare.

La nuova definizione di safety che definiamo è questa:
\begin{align*}
    \vdash t:safe\triangleq \forall k,t'.\IF t\searrow_k t'\THEN \vdash t'.val
\end{align*}
\begin{itemize}
    \item $t\Downarrow$ 
    \item $t\not\to$
    \item $t\Uparrow$
\end{itemize}
La correttezza di questo viene dimostrata per induzione sulla $k$

\section{System F with recursive types}
\begin{align*}
    \V[\tau]^\delta.\tau\times\delta\times v\times n
\end{align*}
Ci aspettiamo che nella value relationship compaia un numero $n$, che indica per quanti step il termine è safe.

Modifichiamo $Semty$ così:
\begin{gather*}
    Semty(\tau)=\{s|s\in\mathcal{P}(\N\times CVal(\tau)),\forall(k,v)\in S,\forall y<k,(j,v)\in S\}
\end{gather*}

\begin{align*}
    & \mathcal{D}[\cdot] = \te{unchanged} \\
    & G[\Gamma,x:\tau]^\delta = \{(k,\gamma[v/x])|(k,\gamma)\in G[\Gamma]^\delta, (k,v)\in \V[\tau]^\delta\} \\
    & \V[\alpha]^\delta = \sigma(\alpha).S \\
    & \V[\N]^\delta = \{(k,n)\} \\
    & \V[\tau_1\to\tau_2]^\delta = \{(k,\lam{x:\tau_1}{t})| \forall j\leq k \forall v \IF (j,v)\in\V[\tau]^\delta \THEN (j,t[v/x])\in\cE[\tau_2]^\delta)\} \\
    & \V[\mu\alpha.\tau]^\delta = \{(k,fold_{\mu\alpha.\tau}v)|\forall j<k (j,v)\in\V[\tau[\mu\alpha.\tau/\alpha]]^\delta \} \\
    & \cE[\tau]^\delta = \{(k,t)|\forall j<k,\forall t' \IF t\searrow_j t' \THEN (k-j,t')\in\V[\tau]^\delta\}
\end{align*}

\begin{gather*}
    \Delta,\Gamma\vDash t.\tau\triangleq \forall \sigma\in D[\Delta],forall(k,\gamma)\in G[\Gamma]^\delta, (k,t\gamma\delta)\in \cE[\tau]^\delta
\end{gather*}

\section{state}
$\LET f=\LET ctr=0 \TIN \lam{x:\N}{\pair{x*2,ctr+1}\TIN f\ 1;f\ 2}$ \\
Fino ad ora questo riduceva a $\pair{4,1}$, perché non abbiamo stato.

\subsection{Adding Heap}
\begin{align*}
    H ::= \emptyset | H,l\mapsto v \\
    \Omega ::= H \triangleright t \\
    t ::= \cdots | new\ t | !t | t:= t \\
    \mE ::= \cdots | new\ \mE | !\mE | \mE:=t | l:=\mE
\end{align*}
Dove $l\in\mathcal{L}$ è la "location", su cui non possiamo però fare pointer arithmetics o simili

\begin{gather*}
    \st[ctx]{
        H\triangleright E[t]\leadsto H'\triangleright E[t']
    }{
        H\triangleright t \leadsto^p H'\triangleright t'
    }
    \\
    \st[prm]{
        H\triangleright \leadsto^p 
    }{
    }
\end{gather*}
nuove regole 
\begin{gather*}
    \st[nbv]{
        H\triangleright new\ v \leadsto^p H;l\mapsto v \triangleright l
    }{
        fresh(l,H)
    }
    \\
    \st[read]{
        H\triangleright !l \leadsto^p H\triangleright v
    }{
        H(l)=v
    }
    \\
    \textcolor{gray}{
        \st[]{
            H,l\mapsto v(l) = v
        }{}
    }\\
    \textcolor{gray}{
        \st[]{
            H,l\mapsto v(l') = v
        }{
            H(l') = v
        }
    }
    \\
    \st[write]{
        H\triangleright l_1:= v l\leadsto^p H'\triangleright \textcolor{blue}{0}
    }{
        H'=H[l_1\mapsto v / l_1\mapsto\_]
    }
\end{gather*}
Dove \textcolor{blue}{0} è una costante generica. Potremmo usare \textit{nil} o qualsiasi altra cosa.

\section{higher order heap}
\begin{gather*}
    \tau ::= \cdots | ref\ \tau \\
    \st[t-new]{
        \Delta,\Gamma\vdash new\ t:\ ref \tau
    }{
        \Delta,\Gamma\vdash t:\tau
    } \\
    \st[t-read]{
        \Delta,\Gamma\vdash !t: \tau
    }{
        \Delta,\Gamma\vdash t:ref \tau
    } \\
    \st[t-read]{
        \Delta,\Gamma\vdash t:=t':\N
    }{
        \Delta,\Gamma\vdash t:ref\ \tau \quad \Delta,\Gamma\vdash t:\tau
    }
\end{gather*}

Aggiungiamo $\Sigma$ per tenere traccia dei type bindings:
\begin{gather*}
    \Sigma ::= \emptyset | \Sigma,l:ref\ \tau \\
    \Delta,\Gamma\vdash t:\tau \to \Sigma,\Delta,\Gamma \vdash t:\tau
\end{gather*}
Modifichiamo progress judgment
\begin{gather*}
    \textcolor{red}{\IF H:\Sigma\AND \Sigma},\emptyset,\emptyset\vdash t:\tau\THEN\ either \vdash t.val\OR \textcolor{red}{\exists t'.H':H\triangleright t\leadsto H'\triangleright t'}
\end{gather*}
Modifichiamo preservation:
\begin{multline*}
    \textcolor{red}{\IF H:\Sigma \AND \Sigma,}\emptyset,\emptyset\vdash t:\tau \AND H\triangleright t\leadsto H'\triangleright t' \\ \textcolor{red}{\THEN \exists\Sigma'\supseteq \Sigma. H':\Sigma'\AND \Sigma',\emptyset,\emptyset\vdash t':\tau}
\end{multline*}
dove:
\begin{gather*}
    \st[]{
        H:\Sigma
    }{
        \forall l:ref\ \tau\in\Sigma. \Sigma,\emptyset,\emptyset\vdash H(l):\tau
    } \\
    \st[]{
        \Sigma,\Delta,\Gamma\vdash l:ref\ \tau
    }{
        \Sigma(l)=ref\ \tau
    }
\end{gather*}


\end{document}
