\documentclass{article}
\usepackage[top=2cm,bottom=2cm,left=2cm,right=3.5cm]{geometry}
\usepackage[utf8]{inputenc}
\usepackage[most]{tcolorbox}
\tcbuselibrary{skins,breakable}

\newtcolorbox{callout}[2][]{breakable,sharp corners, skin=enhancedmiddle jigsaw,parbox=false,
boxrule=0mm,leftrule=2mm,boxsep=0mm,arc=0mm,outer arc=0mm,attach title to upper,
after title={.\ }, coltitle=purple,colback=purple!10,colframe=purple, title={#2},
fonttitle=\bfseries,#1}

\newtcolorbox{warning}[2][]{breakable,sharp corners, skin=enhancedmiddle jigsaw,parbox=false,
boxrule=0mm,leftrule=2mm,boxsep=0mm,arc=0mm,outer arc=0mm,attach title to upper,
after title={!\ }, coltitle=red,colback=red!10,colframe=red, title={#2},
fonttitle=\bfseries,#1}

\newtcolorbox{esempio}[2][]{breakable,sharp corners, skin=enhancedmiddle jigsaw,parbox=false,
boxrule=0mm,leftrule=2mm,boxsep=0mm,arc=0mm,outer arc=0mm,attach title to upper,
after title={:\ }, coltitle=blue,colback=blue!10,colframe=blue, title={#2},
fonttitle=\bfseries,#1}


\usepackage[parfill]{parskip}

\usepackage{graphicx}
\usepackage{algorithm2e}

\title{Robot planning and its applications}
\author{Diego Oniarti}
\date{Anno 2025-2026}

\begin{document}

\maketitle
\tableofcontents

\section{10-09-2025}
The deliberative paradigm is a top-down philosophy where the robot:
\begin{itemize}
    \item Senses the world
    \item plans a course of action
    \item implements the plan
\end{itemize}
This paradigm, for as simple as it is, makes some assumptions. It assumes for 
example that the environment does not change while the plan is being implemented.

The reactive paradigm simply senses the state of the system and performs an action
informed by it. This is a much more simplistic approach on a technical level but
has more limited application. Complex actions are formed by emerging behaviours.

The hybrid paradigm makes a mix of both. It creates a long term plan (deliberate)
and reacts to the environment in real time (reactive) while performing it.

\end{document}
