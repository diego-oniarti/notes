\documentclass{article}
\usepackage[top=2cm,bottom=2cm,left=2cm,right=3.5cm]{geometry}
\usepackage[utf8]{inputenc}
\usepackage[most]{tcolorbox}
\tcbuselibrary{skins,breakable}

\newtcolorbox{callout}[2][]{breakable,sharp corners, skin=enhancedmiddle jigsaw,parbox=false,
boxrule=0mm,leftrule=2mm,boxsep=0mm,arc=0mm,outer arc=0mm,attach title to upper,
after title={.\ }, coltitle=purple,colback=purple!10,colframe=purple, title={#2},
fonttitle=\bfseries,#1}

\newtcolorbox{warning}[2][]{breakable,sharp corners, skin=enhancedmiddle jigsaw,parbox=false,
boxrule=0mm,leftrule=2mm,boxsep=0mm,arc=0mm,outer arc=0mm,attach title to upper,
after title={!\ }, coltitle=red,colback=red!10,colframe=red, title={#2},
fonttitle=\bfseries,#1}

\newtcolorbox{esempio}[2][]{breakable,sharp corners, skin=enhancedmiddle jigsaw,parbox=false,
boxrule=0mm,leftrule=2mm,boxsep=0mm,arc=0mm,outer arc=0mm,attach title to upper,
after title={:\ }, coltitle=blue,colback=blue!10,colframe=blue, title={#2},
fonttitle=\bfseries,#1}


\usepackage[parfill]{parskip}

\usepackage{graphicx}
\usepackage{algorithm2e}

\title{Collaborative Robotics}
\author{Diego Oniarti}
\date{Anno 2025-2026}

\begin{document}

\maketitle
\tableofcontents

\section{2025-09-10}
What is a robot? A programmable machine that can carry out a complex series of
actions automatically.

A robot can sense the \textit{environment}, make \textit{autonomous decisions}, and \textit{operate on the environment}.

\section{2025-09-11}
Robots have \textit{links} and \textit{joints}. Each one has a position and an orientation.

\begin{callout}{Orientation}
    The orientation can be expressed with a rotation matrix. But they are inefficient because they use too many variables. The matrix has $9$ values but only $3$ should really be needed.

    Another way of describing orientation is through \textit{roll, pitch, and yaw}. This is also an intuitive representation but has the \textit{gimble lock} problem.

    Quaternions are straight up magic but they work.
\end{callout}

\paragraph{Direct Kinematics}
To describe the motion of each manipulator link, it is possible to attach a moving frame to each link and describe it w.r.t a fixed reference frame.

The most important frames are 
\begin{itemize}
    \item base frame. This is the one at the base of a manipulator, which never moves
    \item end-effector frame. The "hand" at the end of the manipulator.
\end{itemize}

\paragraph{Workspace}
The workspace of a manipulator is denied as the are in Cartesian space that the end effector can reach.\\
There may also be positions the manipulator can reach with the effector in some orientations but not others

\paragraph{Joints}
A joint $q$ is a connection between two or more links, which allows some motion between the connected links.

Each joint constrains one link to move only in certain directions with respect to the other link. Each possible direction is a \textit{degree of freedom} (DoF)

\begin{callout}{Dof}
    Most actuators have at least $6$ degrees of freedom. This allows them to move objects in any direction and rotation.

    Additional degrees of freedom are said \textit{redundant}, and they increase the dexterity of the robot. This allows the manipulator to do smoother motions and have better kinematics.
\end{callout}

There are many joint types, differentiated by the way the two links can move with respect to one another. Joints may also have an \textit{actuator} (motor) or not.

\paragraph{Effectors}
All different manipulators have and end-effector or a tool of some sort attached to the "wrist" of the arm.\\
It is good to build three revolute joints in the wrist, giving it good dexterity.

\paragraph{Kinematics}
\begin{itemize}
    \item Forward kinematics: find the end-effector pose $x_e$ as a function of the joint configuration $q$
    \item inverse kinematics: find the joint configuration $q$ as a function of the "desired" end-effector pose $x_e$
    \item differential kinematics: Instead of configurations and poses, it links the joint velocity $\dot q$ and position $q$ with the velocity of the end effector.
\end{itemize}

Forward kinematics are not always linear. So the inverse kinematics might and often do have multiple possible solutions.

\paragraph{Dynamics}
is the general problem of describing the motion of a robotic structure as a function of the forces and moments acting on it.\\
In the case of direct dynamics we determine the accelerations and forces acting on the joints and their effects.\\
For inverse dynamics we determine the force the actuators need to apply to get the desired motion.

\paragraph{ISO/TS 15066 Collaborative robot} is a technical specification that regulates
\begin{itemize}
    \item Design of collaborative workspace
    \item Design of collaborative operations
        \begin{itemize}
            \item minimum separation distance 
            \item maximum velocity
            \item etc
        \end{itemize}
    \item methods of collaborative working
        \begin{itemize}
            \item safety-rated monitored sto
            \item hand-guiding
            \item speed and separation
        \end{itemize}
    \item changing between collaborative and non-collaborative actions and different methods of collaboration
\end{itemize}

\paragraph{Compliant Controllers}
A \textit{compliant} (or \textit{admittance}) robot is one that can be manipulated by hand by an human operator/actor. They still try to do their task, but can be grabbed and moved out of the way whenever it is needed.\\
A compliant controller needs sensors to measure forces and torques applied to the joints. Also impedance on the motors can be used.

\section{2025-10-02}
To define a node in \texttt{ros2} we inherit from \texttt{rclcpp::Node}.

rclcpp functions
\begin{enumerate}
    \item shutdown: Wait for Ctrl-C from terminal
    \item Node::create\_publisher<msg\_type>(topic\_name, queue\_size)
    \item Node::create\_wall\_timer(delay, callback) is like setInterval in js.
\end{enumerate}

\section{2025-10-08}

\end{document}
