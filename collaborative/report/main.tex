\documentclass{article}
\usepackage[top=2cm,bottom=2cm,left=2cm,right=3.5cm]{geometry}
\usepackage[utf8]{inputenc}
\usepackage[parfill]{parskip}

\usepackage{graphicx}
\usepackage{subcaption}

\usepackage{hyperref}


\title{Software Development for Collaborative Robotics}
\author{Luca Dematté - Diego Oniarti}
\date{18/12/2025}

\begin{document}

\maketitle
\tableofcontents

\section{Introduction and selected topic}
Our idea originates from a real world scenario that is close to us. \textbf{Fig.\ref{fig:lathe_picture}} shows the lathe present in one of our fathers' workshop. \\
The lathe is equipped with an automatic bar feeder which pushes the material in to
be processed and a tray to store the brass bars. The bars need to be periodically loaded
into the bar feeder by hand. This task is usually performed by two operators since
the bars are approximately \textit{3m} in length and any bend in the material could
render them unusable.

Our project proposal is a robot that would aid an human operator in the task of
loading these rods into the machine. The human will lift the rods from one end while
the robot arm holds the other.

\section{Main Features}
In this section we will describe in more detail the order of operations. The actions
are illustrated visually in \textbf{Fig. \ref{fig:illustrations}}

\paragraph{Activation} The robot should stand in an idle state whenever the
human is not in the proximity of the bar tray. It should then activate when
it detects the human approaching the tray and showing intent to load the feeder.

\paragraph{Recognizing the human's intent} The robot should use its camera to
recognize when the human is starting to lift one end of the bars (which will be taken
an handful at a time) and grab the other.

\paragraph{Matching the human's movements} The robot should use its torque and force
sensors to respond to the human's movements, moving accordingly to keep the rods level.

\paragraph{Releasing the bars} The robot should be able to recognize when the human
is releasing the bars into the feeder and do the same before returning to an idle position.

\section{Software components}
\paragraph{Control} We will use an impedance/admittance controller from the FZI
repository to control the movement of the robot.

\paragraph{Interaction} To second the human movement we will simply monitor the force
and the torque on the end effector and operate accordingly.

\paragraph{Safety} As stated in the report the robot will have power and force limiting,
Speed and separation monitoring, and an emergency stop.

\paragraph{Task Planning} The task the robot should perform is easily divisible into
discrete phases that should be performed in sequence, so we believe the best approach
would be a finite state machine. Hence we consider the use of the yasmin library.

\paragraph{Motion planning} The robot would use simple spline interpolation to reach
the bars and then switch to compliance.

\begin{figure}[ht!]
    \center
    \includegraphics[width=0.8\linewidth]{images/tornio.jpg}
    \caption{The lathe and the brass bars}
    \label{fig:lathe_picture}
\end{figure}

\begin{figure}[htbp]
    \centering
    \begin{subfigure}{0.45\textwidth}
        \centering
        \includegraphics[width=\linewidth]{images/idle.jpg}
        \caption{Robot idle}
        \label{fig:idle}
    \end{subfigure}\hfill
    \begin{subfigure}{0.45\textwidth}
        \centering
        \includegraphics[width=\linewidth]{images/recognize.jpg}
        \caption{Robot recognizing the human intent}
        \label{fig:recognition}
    \end{subfigure}
    \medskip
    \begin{subfigure}{0.45\textwidth}
        \centering
        \includegraphics[width=\linewidth]{images/follow.jpg}
        \caption{Robot following the human}
        \label{fig:follow}
    \end{subfigure}\hfill
    \begin{subfigure}{0.45\textwidth}
        \centering
        \includegraphics[width=\linewidth]{images/load.jpg}
        \caption{Robot and human loading the feeder}
        \label{fig:load}
    \end{subfigure}

    \caption{Main features}
    \label{fig:illustrations}
\end{figure}

\end{document}
