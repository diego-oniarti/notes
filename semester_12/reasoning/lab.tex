\section{LAB 1}
\subsection{DIMACS}
Dimacs is the standard format for representing SAT problems.
\paragraph{File structure}
\begin{itemize}
    \item Comments start with 'c'
    \item header of the form 'p cnf 7 8'
        \begin{itemize}
            \item 7: number of variables
            \item 8: number of clauses
        \end{itemize}
    \item 1 clause for line of form "1 -2 0"
        \begin{itemize}
            \item trailing 0 is a constant
            \item 1 first variable
            \item -2 second variable, negated
        \end{itemize}
\end{itemize}

\paragraph{command}
\texttt{mathsat -input=dimacs -model file.cnf}\\
\texttt{-model} tells it to provide a model, otherwise it gives a yes/no answer

\section{19/03/202}
(set-option :produce-models true) Generation of models \\
(set-option :produce-unsat-cores true) Extraction of UNSAT cores \\
(set-option :produce-proofs true) Building UNSAT proof \\
(set-logic < logic >) Set background logic \\


